\documentclass{beamer}
\usepackage[T1]{fontenc}
\usepackage[utf8]{inputenc}
\usepackage{listings}
\usepackage{amssymb}
\usepackage{upgreek}
\usepackage{mathtools}

\usepackage{color}
\definecolor{keywordcolor}{rgb}{0.7, 0.1, 0.1}   % red
\definecolor{commentcolor}{rgb}{0.4, 0.4, 0.4}   % grey
\definecolor{symbolcolor}{rgb}{0.0, 0.1, 0.6}    % blue
\definecolor{sortcolor}{rgb}{0.1, 0.5, 0.1}      % green
\definecolor{errorcolor}{rgb}{1, 0, 0}           % bright red
\definecolor{stringcolor}{rgb}{0.5, 0.3, 0.2}    % brown

\def\lstlanguagefiles{lstlean.tex}
% set default language
\lstset{language=lean}

\usetheme{Madrid}
\usecolortheme{default}

\newcommand{\N}{\mathbb{N}}
\newcommand{\R}{\mathbb{R}}
\newcommand{\interior}[1]{\text{int}(#1)}
\newcommand{\restrict}[2]{\ensuremath{\left.#1\right|_{#2}}}
\newcommand{\boundary}{\partial}

%Information to be included in the title page:
\title{The product of CW-complexes}
\author{Hannah Scholz}
\institute[MI]{Mathematical Institute of the University of Bonn}
\date{22.07.2024}



\begin{document}

\frame{\titlepage}

\begin{frame}
\frametitle{Reminder: Definition of CW-complexes}
\begin{block}{Definition: CW-complex}
    Let $X$ be a Hausdorff space. 
    A \alert{CW-structure} on $X$ consists of a family of indexing sets $(I_n)_{n \in \N}$ and a family of maps $(Q_i^n\colon D_i^n\rightarrow X)_{n \ge 0, i \in I_n}$ s.t.
    \setbeamertemplate{enumerate items}[default]
    \begin{enumerate}[(i)]
        \item $\restrict{Q_i^n}{\interior{D_i^n}}\colon \interior{D_i^n} \rightarrow Q_i^n(\interior{D_i^n})$ is a homeomorphism. We call $e_i^n \coloneq Q_i^n(\interior{D_i^n})$ an \alert{$n$-cell} (or a cell of dimension $n$).
        \item For all $m, n \in \N$, $i \in I_n$ and $j \in I_m$, $Q_i^n(\interior{D_i^n})$ and $Q_j^m(\interior{D_j^m})$ are disjoint.
        \item For each $n \in \N$, $i \in I_n$, $Q_i^n(\boundary D_i^n)$ is contained in the union of a finite number of cells of dimension less than $n$.
        \item $A \subseteq X$ is closed iff $Q_i^n(D_i^n) \cap A$ is closed for all $n \in \N$ and $i \in I_n$.
        \item $\bigcup_{n \ge 0}\bigcup_{i \in I_n}e_i^n = X$.
    \end{enumerate}
\end{block}
\end{frame}

\begin{frame}[fragile]
\frametitle{Lean: Definition of CW-Complexes}
    \begin{exampleblock}{Lean: CW-complex}
    \begin{lstlisting}[basicstyle=\ttfamily\footnotesize]
class CWComplex.{u} {X : Type u} [TopologicalSpace X] (C : Set X) where
  cell (n : ℕ) : Type u
  map (n : ℕ) (i : cell n) : PartialEquiv (Fin n → ℝ) X
  source_eq (n : ℕ) (i : cell n) : (map n i).source = closedBall 0 1
  cont (n : ℕ) (i : cell n) : ContinuousOn (map n i) (closedBall 0 1)
  cont_symm (n : ℕ) (i : cell n) : ContinuousOn (map n i).symm (map n i).target
  pairwiseDisjoint :
    (univ : Set (Σ n, cell n)).PairwiseDisjoint (fun ni ↦ map ni.1 ni.2 '' ball 0 1)
  mapsto (n : ℕ) (i : cell n) : ∃ I : Π m, Finset (cell m),
    MapsTo (map n i) (sphere 0 1) (⋃ (m < n) (j ∈ I m), map m j '' closedBall 0 1)
  closed (A : Set X) (asubc : A ⊆ C) : IsClosed A ↔ ∀ n j, IsClosed (A ∩ map n j '' closedBall 0 1)
  union : ⋃ (n : ℕ) (j : cell n), map n j '' closedBall 0 1 = C
    \end{lstlisting}
    \end{exampleblock}
\end{frame}

\begin{frame}[fragile]
\frametitle{K-ification}
    \begin{block}{Definition: k-ification}
        Let $X$ be a topological space. 
        Then we can define another topological space $X_c$ on the same set by setting: 
        \[A \subseteq X_c \text{ open} \iff A \cap C \text{ open in } C \text{ for every compact set } C \subseteq X.\]
        We call $X_c$ the \alert{k-ification} of $X$.
    \end{block}
    \begin{exampleblock}{Lean: k-ification}
    \begin{lstlisting}[basicstyle=\ttfamily\footnotesize, belowskip=-0.8 \baselineskip]
def kification (X : Type*) := X

instance instkification {X : Type*} [t : TopologicalSpace X] : TopologicalSpace (kification X) where
  IsOpen A := ∀ (B : t.Compacts), ∃ (C: t.Opens), A ∩ B.1 = C.1 ∩ B.1
  isOpen_univ := ... 
  isOpen_inter := ... 
  isOpen_sUnion := ...
    \end{lstlisting}
    \end{exampleblock}
\end{frame}

\begin{frame}[fragile]
\frametitle{Product of CW-complexes}
    \begin{block}{Theorem (product of CW-complexes)}
        Let $X$, $Y$ be CW-complexes with families of characteristic maps $(Q_i^n \colon D_i^n \to X)_{n, i}$ and $(P_j^m \colon D_j^m \to Y)_{m, j}$. 

        Then we get a CW-structure on $(X \times Y)_c$ with characteristic maps $(Q_i^n \times P_j^m \colon D_i^n \times D_j^m \to (X \times Y)_c)_{n,m,i,j}$.
    \end{block}
    \begin{exampleblock}{Lean: product of CW-complexes}
    \begin{lstlisting}[basicstyle=\ttfamily\footnotesize, belowskip=-0.8 \baselineskip]
def Prodkification X Y := kification (X × Y)
infixr:35 " ×ₖ "  => Prodkification

instance CWComplex_product : CWComplex (X := X ×ₖ Y) (C ×ˢ D) where
  cell n := (Σ' (m : ℕ) (l : ℕ) (hml : m + l = n), cell C m × cell D l)
  map n i := match i with
    | ⟨m, l, hmln, j, k⟩ =>
      hmln ▸ Equiv.transPartialEquiv ((IsometryEquivFinMap m l).symm).toEquiv
      (PartialEquiv.prod (map m j) (map l k))
    \end{lstlisting}
    \end{exampleblock}
\end{frame}

\begin{frame}
\frametitle{Why don't people care about the proof?}
  \begin{itemize}
    \item Answer: Most important spaces are already k-spaces.
  \end{itemize}
  \begin{block}{Lemma (Examples of k-spaces)}
    \setbeamertemplate{enumerate items}[default]
    \begin{enumerate}[(i)]
      \item Weakly locally compact spaces are k-spaces.
      \item Sequential spaces (e.g. first countable spaces, metric spaces) are k-spaces.
    \end{enumerate}
  \end{block}
  \begin{block}{Examples of spaces that are not k-spaces}
    \setbeamertemplate{enumerate items}[default]
    % See https://en.wikipedia.org/wiki/Compactly_generated_space#CITEREFWillard2004
    \begin{enumerate}[(i)]
      \item Every non-discrete anti-compact T$_1$ space (e.g. cocountable topology on uncountable set) is not a k-space.
      \item The product of uncountably many copies of $\R$ is not a k-space.
    \end{enumerate}
  \end{block}
\end{frame}

\begin{frame}
\frametitle{Proof of continuity}
  \begin{block}{Lemma (alternate definition of k-ification)}
    We have
    \[A \subseteq X_c \text{ closed} \iff A \cap C \text{ closed in } C \text{ for every compact set } C \subseteq X.\]

  \end{block}
  \begin{block}{Lemma}
    Let $X$ and $Y$ be topological spaces with $X$ compact. 
    Let $f \colon X \to Y$ be a continuous.
    Then $f \colon X \to Y_c$ is continuous.
  \end{block}
\end{frame}

\begin{frame}
\frametitle{Proof of weak topology}
  \begin{block}{Lemma}
    Let $X$ be a CW-complex and $C \subseteq X$ a compact set. 
    Then $C$ is disjoint with all but finitely many cells of $X$.
  \end{block}
  \begin{block}{Lemma}
    $(X \times Y)_c$ has weak topology 
    i.e. $A \subseteq (X \times Y)_c$ is closed iff $Q_i^n \times P_j^m(D^{n + m}) \cap A$ is closed for all $n, m \in \N$, $i \in I_n$ and $j \in J_m$.
  \end{block}
\end{frame}



\end{document}