\section{Definition of a CW-complexes}

The modern definition of a CW-complex is the following: 

\begin{defi}
    Let $X$ be a topological space. 
    A CW-complex on $X$ is a filtration 
    $X_0 \subseteq X_1 \subseteq X_2 \subseteq \dots$ such that 
    \begin{enumerate}[label=(\roman*)]
        \item For every $n \ge 0$ there is a pushout of topological spaces
            \[
            \begin{tikzcd}[row sep = 1.5cm, column sep = 1.5cm]
                \displaystyle \coprod_{i \in I_n}S_i^{n - 1} \arrow[r, "\coprod_{i \in I_n"}q_i^n"] \arrow[d, hook, "\coprod_{i \in I_n} j_i"'] & X_{n- 1} \arrow[d] \\ 
                \displaystyle \coprod_{i \in I_n}D_i^n \arrow[r, "\coprod_{i \in I_n} Q_i^n"'] & X_n
            \end{tikzcd}
            \] 
            where $I_n$ is any indexing set and $j_i \colon S_i^{n - 1} \to D_i^n$ is the usual inclusion for every $i \in I_n$.
        \item We have $X = \bigcup_{n \ge 0} X_n$. 
        \item $X$ has weak topology, i.e. $A \subseteq X$ is open $\iff A \cap X_n$ is open in $X_n$ for every $n$.
    \end{enumerate}
    $X_n$ is called the n-skeleton. 
    An element $e^n \in \pi_0 (X_n \setminus X_{n - 1})$ is called an (open)$n$-cell. 
    $Q_i^n$ is called a characteristic map.
\end{defi}

In this thesis we will however focus on the historical definition of CW-complexes first presented by \Citeauthor{Whitehead2018} in \cite{Whitehead2018}.

\begin{defi}
    Let $X$ be a Hausdorff space. 
    A CW-structure on $X$ consists of a family of indexing sets $(I_n)_{n \in \bN}$ and a family of maps $(Q_i^n\colon D_i^n\rightarrow X)_{n \ge 0, i \in I_n}$ s.t.
    \begin{enumerate}[label=(\roman*)]
        \item $\restrict{Q_i^n}{\interior{D_i^n}}\colon \interior{D_i^n} \rightarrow Q_i^n(\interior{D_i^n})$ is a homeomorphism. We call $e_i^n \coloneq Q_i^n(\interior{D_i^n})$ an (open) $n$-cell (or a cell of dimension $n$).
        \item For all $n, m \in \bN$, $i \in I_n$ and $j \in I_m$ where $(n, i) \ne (m, j)$ the cells $Q_i^n(\interior{D_i^n})$ and $Q_j^m(\interior{D_j^m})$ are disjoint.
        \item For each $n \in \bN$, $i \in I_n$, $Q_i^n(\boundary D_i^n)$ is contained in the union of a finite number of cells of dimension less than $n$.
        \item $A \subseteq X$ is closed iff $Q_i^n(D_i^n) \cap A$ is closed for all $n \in \bN$ and $i \in I_n$.
        \item $\bigcup_{n \ge 0}\bigcup_{i \in I_n}e_i^n = X$.
    \end{enumerate}
    We call $Q_i^n$ a characteristic map and $\closure{e}_i^n \coloneq Q_i^n(D_i^n)$ a closed $n$-cell for any $i$ and $n$.
\end{defi}