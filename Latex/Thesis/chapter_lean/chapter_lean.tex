\chapter{Lean and mathlib}

In this chapter we will discuss some general concepts about Lean and its mathematical library mathlib. 

Lean itself is a programming language that is frequently used as a theorem prover. 
It was primarily developed by Leonardo de Moura who co-funded the Lean focused research organisation that has taken on the development for five years in 2023 \cite{LeanFRO2024}. 
The currently used version is called Lean 4.
More about the technical details of Lean 4 can be found in \cite{deMoura2021}.

Lean is known for its extensive mathematical library called \emph{mathlib} of which the development has been largely community driven. 
Its github repository has just over 300 different contributors and multiple new pull requests every day that get approved or rejected by the 28 maintainers. 

There have been multiple large formalizations based on mathlib. 
Here are two examples: 
In the \emph{Liquid Tensor Experiment}, given to the Lean community by Peter Scholze as a challenge, Johan Commelin, Adam Topaz and a number of other contributors formalised a theorem by Peter Scholze and Dustin Clausen from condensed mathematics \cite{Commelin2022}.
Floris van Doorn, Patrick Massot and Oliver Nash have formalised the existence of sphere eversions, a concept from differential topology, showing that geometric areas of mathematics can also be successfully formalised in Lean \cite{vanDoorn2023}. 

There are also several large scale ongoing projects of which we again present two examples: 
Floris van Doorn is currently leading a a formalisation of a generalisation of Carleson's theorem, a theorem from fourier analysis, by Christoph Thiele and his group \cite{Becker2024}.
Additionally there is a project led by Kevin Buzzard that aims to reduce the famous Fermat's Last Theorem to mathematical facts already known by mathematicians in the 1980s, a starting point similar to that of Andrew Wiles and Richard Taylor, who first proved this theorem in 1995 \cite{Buzzard2024}. 

\section{The type theory of Lean}
\label{sec:typetheory}

\emph{Type theory} was first proposed by \Citeauthor{Russell1908} in \citeyear{Russell1908} \cite{Russell1908} as a way to axiomatise mathematics and resolve the paradoxes (most famously Russell's paradox) that were discussed at the time. 
While type theory has lost its relevance as a foundation of mathematics to set theory it has since been studied in both mathematics and computer science. 
It was first used in formal mathematics in 1967 in the formal language \emph{AUTOMATH}. 
More about the history of type theory can be found in \cite{Kamareddine2004}. 
Discussions of type theory in mathematics and especially its connections to homotopy theory forming the new area of \emph{homotopy type theory} can be found in \cite{hottbook}.
We will now focus on the type theory as used in Lean.
A detailed account of that topic can be found in \cite{Carneiro2019}. 
The following short discussion is be based on \cite{Avigad2024}. 

Lean uses what is called a \emph{dependent type theory}.
In type theory every object has a type. 
A type can for example be the natural numbers which we write in Lean as \lstinline{Nat} or \lstinline{ℕ} or propositions which is written as \lstinline{Prop}.
To assert that \lstinline{n} is a natural number or that \lstinline{p} is a proposition we write \lstinline{n : ℕ} and \lstinline{p : Prop}. 
Proofs of a proposition \lstinline{p} also form a type written as \lstinline{Proof p}.
If you want to say that \lstinline{hp} is a proof of \lstinline{p} then you can simply write \lstinline{hp : p}.
Something to note about proofs in Lean is that contrary to other type theories the type theory of Lean has \emph{proof irrelevance} which means that two proofs of a proposition \lstinline{p} are by definition assumed to be the same.

Even types themselves are types. 
In Lean the type of natural numbers \lstinline{ℕ} has the type \lstinline{Type}. 
The type of propositions \lstinline{Prop} is also of type \lstinline{Type}.
As to not run into Russel's paradox there is a hierarchy of types. 
The type of \lstinline{Type} is \lstinline{Type 1}, the type of \lstinline{Type 1} is \lstinline{Type 2} and so on.
These are called type-universes. 
The notation \lstinline{α : Type*} is a way of stating that \lstinline{α} is a type in an arbitrary universe.

There are a few ways to construct new types from existing ones. 
Some of them are very similar to constructions on sets such as the cartesian product of types written as \lstinline{α × β} or the type of functions from \lstinline{α} to \lstinline{β} written as \lstinline{α → β} where \lstinline{α} and \lstinline{β} are types.
Since these are quite self-explanatory we will not go into more detail.
We will now mainly discuss constructions that do not fulfil this criterium. 
A first example is the sum type of two types \lstinline{α} and \lstinline{β} written as \lstinline{α ⊕ β} which is the equivalent to a disjoint union of sets. 

The next two examples explain why this type theory is a dependent type theory:
If we have a type \lstinline{α} and for every \lstinline{a : α} a type \lstinline{β a} (i.e. \lstinline{β} is a function assigning a type to every \lstinline{a : α}) then we can construct the pi type or dependent function type written as \lstinline{(a : α) → β} or \lstinline{Π (a : α), β a}. 
The dependent version of the cartesian product is called a sigma type and can be written as \lstinline{(a : α) × β a} or \lstinline{Σ a : α, β a} for \lstinline{α} and \lstinline{β} the same way as above.

\section{Implicit variables and type class inference}

A crucial factor that makes lean more comfortable to use and makes the formalisation process feel closer to doing mathematics on paper is its use of \emph{implicit variables} and \emph{typeclass inference}. 
We will explain both of these concepts in this section. 

First let us discuss implicit variables based on \cite{Avigad2024}. 
One way that we could define continuity in lean is the following: 

\begin{lstlisting}
structure Continuous' (X Y : Type*) (t : TopologicalSpace X) 
    (s : TopologicalSpace Y) (f : X → Y) : Prop where
  isOpen_preimage : ∀ s, IsOpen s → IsOpen (f ⁻¹' s)
\end{lstlisting}

But now if we are given two types \lstinline{X} and \lstinline{Y} with topologies \lstinline{s} and \lstinline{t} respectively and a map \lstinline{f : X → Y}, the statement that the map \lstinline{f} is continuous would be expressed in the following way: 

\begin{lstlisting}
    Continuous' X Y t s f
\end{lstlisting}

which is a lot longer than we would write this on paper. 

One thing that we can notice is that the types \lstinline{X} and \lstinline{Y} are contained in the definition of \lstinline{f} which means that lean should be able to find that information itself. 
To tell lean to do that you can replace the variables by underscores: 

\begin{lstlisting}
Continuous' _ _ t s f
\end{lstlisting}

Since this should always possible to be inferred we can change the definition to already specify that the two types should be clear from the context. 
We use curly brackets to do this: 

\begin{lstlisting}
structure Continuous'' {X Y : Type*} (t : TopologicalSpace X) 
    (s : TopologicalSpace Y) (f : X → Y) : Prop where
  isOpen_preimage : ∀ s, IsOpen s → IsOpen (f ⁻¹' s)
\end{lstlisting}

which enables us to write continuity like this: 

\begin{lstlisting}
    Continuous'' t s f
\end{lstlisting}

This is already a lot shorter than what we had above but there is still room for improvement as on paper you would probably just write "$f$ is continuous" since in most contexts $X$ and $Y$ will only have one specified topology each that can be inferred by the reader. 
The same thing is also true in lean and we can achieve this by typeclass inference.
Typeclasses were first invented by \Citeauthor{Wadler1989} in \cite{Wadler1989} to be used in the programming language Haskell. 
They are a way to overload operations for various different types. 
For example you might want to write code that works for all types that have a topology. 
In lean this is possible by just stating that you input type \lstinline{X} is part of the typeclass \lstinline{TopologicalSpace}. 
You can specify that something ia a typeclass with the keyword \lstinline{class}. 
The definition of the typeclass of topological spaces in mathlib looks like this

\begin{lstlisting}
class TopologicalSpace (X : Type*) where
  protected IsOpen : Set X → Prop
  protected isOpen_univ : IsOpen univ
  protected isOpen_inter : ∀ s t, IsOpen s → IsOpen t → IsOpen (s ∩ t)
  protected isOpen_sUnion : ∀ s, (∀ t ∈ s, IsOpen t) → IsOpen (⋃₀ s)
\end{lstlisting}

Let us first explain what this code means: 
The keyword \lstinline{protected} means that these properties should not be accessed directly because there are lemmas that should be used instead. 
\lstinline{Set X} is the type that consists of all sets of elements of \lstinline{X}.
Thus the line \lstinline{protected IsOpen : Set X → Prop} expresses that \lstinline{IsOpen} is a property that can be assigned to a set in \lstinline{X}.
The rest of the lines discuss the properties of a topology.
\lstinline{univ} is the set that is composed of all elements of \lstinline{X} and \lstinline{⋃₀ s} is the union over the set \lstinline{s}. 
All of these explanations are not actually relevant to typclasses, they are just for your understanding of the above code. 

Typeclasses are also expected to be inferred automatically. 
Local instances of these typeclasses can written with square brackets which tells lean to infer these automatically.

We can now look at the version of continuity that is almost identical to that of mathlib: 

\begin{lstlisting}
structure Continuous {X Y : Type*} [t : TopologicalSpace X]
    [s : TopologicalSpace Y] (f : X → Y) : Prop where
  isOpen_preimage : ∀ s, IsOpen s → IsOpen (f ⁻¹' s)
\end{lstlisting}

which enables us to write that \lstinline{f} is continuous in the context explained above as follows:

\begin{lstlisting}
Continuous f
\end{lstlisting}

When you define a class you can then define instances of that class to be inferred whenever you talk you talk about a type with that instance. 