\section{K-spaces and the k-ification}

Before we can move on to discuss the product of CW-complexes we need to discuss its topology.
Therefore we will study k-spaces and the k-ification in this section.

A k-space or also called a compactly generated space is defined for our purposes as follows. 
Note that we mean quasi-compactness when talking about compactness.

\begin{defi}
    Let $X$ be a topological space. 
    We call $X$ a \emph{k-space} if 
    \[A \subseteq X \text{ is open} \iff \text{for all compact sets } C \subseteq X \text{ the intersection } A \cap C \text{ is open in } C.\]
\end{defi}

There are a lot of different definitions in the literature. 
The most popular ones all agree on Hausdorff spaces. 
An overview of these different notions can be found on Wikipedia \cite{Wikipedia2024}.

It will also be helpful to characterise k-spaces via closed sets: 

\begin{lem}\label{lem:closediffinkspace}
    Let $X$ be a topological space.
    $X$ is a k-space iff  
    \[A \subseteq X \text{ is closed} \iff \text{for all compact sets } C \subseteq X \text{ the intersection } A \cap C \text{ is closed in } C.\]
\end{lem}
\begin{proof}
    We only show that forward direction as the backward direction follows in the same way.
    Of the equivalence that we now need to show the forward direction is trivial.
    Thus let $A \subseteq X$ be a set such that for all compact sets $C \subseteq X$ the intersection $A \cap C$ is closed in $C$. 
    It is enough to show that $\compl{A}$ is open. 
    By definition of the k-space that is the case if for every compact set $C \subseteq X$ the intersection $\compl{A} \cap C$ is open in $C$. 
    Take any compact $C \subseteq X$.
    By assumption $A \cap C$ is closed in $C$.
    Since $A \cap C$ is the complement of $\compl{A} \cap C$ in $C$, this immediately gives us that $\compl{A} \cap C$ is open in $C$.
\end{proof}

We also define a way to make any topological space into a k-space which we call the k-ification: 

\begin{defi}
    Let $X$ be a topological space. 
    We can define another topological space $X_c$ on the same set by setting
    \[A \subseteq X_c \text{ is open} \iff \text{for all compact sets } C \subseteq X \text{ the intersection } A \cap C \text{ is open in } C.\]
    We call $X_c$ the \emph{k-ification} of $X$.
\end{defi}

It is easy to see that this gives us a finer topology: 

\begin{lem}\label{lem:finertopology}
    $A \subseteq X$ is open $\implies A \subseteq X_c$ is open.
\end{lem}

Again it it useful to characterise the closed sets in the k-ification: 

\begin{lem}\label{lem:closediffinkification}
    $A \subseteq X_c$ is closed $\iff A \cap C$ is closed in $C$ for all compact sets $C \subseteq X$.
\end{lem}
\begin{proof}
    Completely analogue to the proof of lemma \ref{lem:closediffinkspace}.
\end{proof}

To show that the k-ification actually fulfils its purpose of turning any space into a k-space, we first need the following lemma:

\begin{lem}\label{lem:compactiffcompact}
    $A \subseteq X$ is compact $\iff A \subseteq X_c$ is compact.
\end{lem}
\begin{proof}
    For the backward direction notice that lemma \ref{lem:finertopology} is another way of stating that the map $\id \colon X_c \to X$ is continuous. 
    As the image of a compact set under a continuous map, that makes $A \subseteq X$ compact. 

    For the forward direction take $A \subseteq X$ compact. 
    To show that $A \subseteq X_c$ is compact, take an open cover $(U_i)_{i \in \iota}$ of $A$ in $X_c$. 
    For every $i \in \iota$ there is by definition of the k-ification an open set $V_i \subseteq X$ such that $V_i \cap A = U_i \cap A$.  
    $(V_i)_{i \in \iota}$ is an (open) cover of $A$ in $X$: 
    \[A = A \cap \bigcup_{i \in \iota} U_i = \bigcup_{i \in \iota} (A \cap U_i) = \bigcup_{i \in \iota} (A \cap V_i) = A \cap \bigcup_{i \in \iota} V_i \subseteq \bigcup_{i \in \iota} V_i.\]
    Thus there is a finite subcover $(V_i)_{i \in \iota'}$ of $A$ in $X$.
    $(U_i)_{i \in \iota'}$ is now a finite subcover of $A$ in $X_c$: 
    \[A = A \cap \bigcup_{i \in \iota'} V_i = \bigcup_{i \in \iota'} (A \cap V_i) = \bigcup_{i \in \iota'} (A \cap U_i) = A \cap \bigcup_{i \in \iota'} U_i \subseteq \bigcup_{i \in \iota'} U_i.\]
\end{proof}

Now we are ready to move on to the promised lemma:

\begin{lem}
    $X_c$ is a k-space for every topological space $X$.
\end{lem}
\begin{proof}
    We need to show that a set $A \subseteq X_c$ is open iff $A \cap C$ is open in $C$ for every compact set $C \subseteq X_c$. The forward direction is again trivial. 
    
    For the backward direction take a set $A \subseteq X_c$ such that for every compact set $C \subseteq X_c$ the intersection $A \cap C$ is open in $C$. 
    By the definition of the k-ification it is enough to show that for every compact set $C \subseteq X$ the intersection $A \cap C$ is open in $C$. So let $C \subseteq X$ be a compact set. 
    By \ref{lem:compactiffcompact} $C$ is also compact in $X$.
    By assumption this means that $A \cap C$ is open in $C \subseteq X_c$ (in the subspace topology of the k-ification). Thus there is an open set $B \subset X_c$ such that $A \cap C = B \cap C$. 
    By the definition of the k-ification $B \cap C$ is open in $C \subseteq X$. 
    That means there is an open set $E \subseteq X$ such that $B \cap C = E \cap C$. 
    But that now gives us $A \cap C = B \cap C = E \cap C$ with which we can conclude that $A \cap C$ is open in $C \subseteq X$ (in the subspace topology of the original topology of X).
\end{proof}

If we already have a k-space, then the k-ification just maintains the topology of our space:

\begin{lem}\label{lem:kificationkspace}
    Let $X$ be a k-space.
    Then the topologies of $X$ and $X_c$ coincide.
\end{lem}
\begin{proof}
    Notice that the characterisation of open sets in $X$ and $X_c$ respectively agree in this setting.
\end{proof}

\begin{cor}
    The k-ification is idempotent.
\end{cor}

Now we will characterise continuous maps to and from the k-ification. 
Going from the k-ification is not a big issue: 

\begin{lem}
    Let $f \colon X \to Y$ be a continuous map of topological spaces.
    Then $f \colon X_c \to Y$ is continuous.
\end{lem}
\begin{proof}
    This follows easily from lemma \ref{lem:finertopology}.
\end{proof}

More interesting questions are when a map to the k-ification or a map from a k-ification to a k-ification is continuous. 
The following two lemmas and proofs that answer these questions are based on lemma 46.4 of \cite{Munkres2014}.
The next lemma is the first step towards the answer:

\begin{lem}
    Let $X$ be a compact space and $f \colon X \to Y$ be a continuous map. 
    Then $f \colon X \to Y_c$ is continuous.
\end{lem}
\begin{proof}
    We want to show that for every closed $A \subseteq Y_c$ the preimage $\preim{f}{A}$ is closed in $X$.
    Take any closed set $A \subseteq Y_c$. 
    We know by lemma \ref{lem:closediffinkification} that $A \cap C$ is closed in $C$ for every compact $C \subseteq Y$.
    As the image of a compact set $f(X)$ is compact. 
    Thus $A \cap f(X)$ is closed in $f(X) \subseteq Y$. 
    By the definition of the subspace topology there is a closed set $B \subseteq Y$ such that $A \cap f(X) = B \cap f(x)$. 
    Now we have 
    \[\preim{f}{A} = \preim{f}{A \cap f(X)} = \preim{f}{B \cap f(X)} = \preim{f}{B}\]
    which is closed as the preimage of a closed set under a continuous map.
\end{proof}

Now this helps us get the following lemma:

\begin{lem}
    Let $f \colon X \to Y$ be a map of topological spaces such that for every compact $C \subseteq X$ the restriction $\restrict{f}{C} \colon C \to Y$ is continuous.
    Then $f \colon X_c \to Y_c$ is continuous.
\end{lem}
\begin{proof}
    The last lemma together with our assumption tells us that for every compact $C \subseteq X$ the restriction $\restrict{f}{C} \colon C \to Y_c$ is continuous.
    To show the claim take any open $A \subseteq Y_c$.
    We need to show that $\preim{f}{A} \subseteq X_c$ is open. 
    By definition of the k-ification this set is open if for all compact sets $C \subseteq X$ the intersection $\preim{f}{A} \cap C$ is open in $C$. 
    Take any compact set $C \subseteq X$. 
    As noted above we now know that $\restrict{f}{C} \colon C \to Y_c$ is continuous.
    Or in other words we know that for every open $B \subseteq Y_c$ there is an open set $E \subseteq X$ such that $\preim{f}{B} \cap C = E \cap C$.
    Applying this to the set $A \subseteq Y_c$ gives us an open set $E \subseteq X$ such that $\preim{f}{A} \cap C = E \cap C$.
    But that is just another way of stating that $\preim{f}{A}$ is open in $C \subseteq X$.
\end{proof}

That yields the following corollary: 

\begin{cor}
    Let $f \colon X \to Y$ be a continuous map of topological spaces.
    Then $f \colon X_c \to Y_c$ is continuous.
\end{cor}
\begin{proof}
    This situation trivially fulfils the conditions of the previous lemma.
\end{proof}

If you look at the discussion of the product of CW-complexes in some topology books, for example \cite{Hatcher2001} and \cite{Lück2005}, you will notice that the k-ification rarely gets discussed in detail. 
One possible reason for this is that most common spaces that you encounter are already k-spaces.
Lemma \ref{lem:kificationkspace} then allows you to ignore the k-ification entirely. 
We will therefore discuss in the remainder of this section which spaces are k-spaces and which are not.
The first example are weakly locally compact spaces.

\begin{defi}
    Let $X$ be a topological space.
    We call $X$ \emph{weakly locally compact} if every point $x \in X$ has some compact neighbourhood.
\end{defi}

This property is in some sources just called locally compact.
The following proof is from Lemma 46.3 in \cite{Munkres2014}.

\begin{lem}
    Weakly locally compact spaces are k-spaces.
\end{lem}
\begin{proof}
    Let $X$ be a weakly locally compact space. 
    Let $A \subseteq X$.
    We need to show that $A$ is open iff $A \cap C$ is open in $C$ for every compact set $C$.
    The forward direction is trivial. 
    So assume that that for every compact set $C$ the intersection $A \cap C$ is open in $C$.
    $A$ is open if it is a neighbourhood of every point $x \in A$.
    So fix any $x \in A$.
    Since $X$ is weakly locally compact, $x$ has a compact neighbourhood $C$.
    By definition of neighbourhoods there is an open set $U \subseteq C$ such that $x \in U$ and we need to find an open set $V \subseteq A$ such that $x \in V$.
    We show that $A \cap U$ fulfils these conditions. 
    It is obvious that $A \cap U \subseteq A$ and $x \in A \cap U$. 
    So it is left to show that $A \cap U$ is open.
    By assumption $A \cap C$ is open in $C$. 
    That means that there is an open set $B$ such that $A \cap C = B \cap C$. 
    This now gives us 
    \[A \cap U = A \cap C \cap U = B \cap C \cap U = B \cap U\]
    which is open as the intersection of two open sets. 
\end{proof}

Another big class of spaces which are k-spaces are sequential spaces. 

\begin{defi}
    A set $A$ in a topological space $X$ is \emph{sequentially closed} if for every convergent sequence contained in $A$ its limit point is also in $A$. 
    The \emph{sequential closure} of a set $A$ in $X$ is defined as $\seqclosure{A} = \{ x \in X \mid \text{there is a sequence } (a_n)_{n \in \bN} \subseteq A \text{ such that} (a_n)_{n \in \bN} \text{ converges to } x\}$.
    A \emph{sequential space} is a space in which all sequentially closed sets are closed.
\end{defi}

We will need the following characterisation of sequentially closed sets: 

\begin{lem}
    A set $A \subseteq X$ is sequentially closed iff $A = \seqclosure{A}$.
\end{lem}
\begin{proof}
    This is easy to see from the definitions.
\end{proof}

The following proof is based on \cite{Scott2016} and Lemma 46.3 in \cite{Munkres2014}. 

\begin{lem}
    Sequential Spaces are k-spaces.
\end{lem}
\begin{proof}
    Let $X$ be a Sequential Space. 
    By lemma \ref{lem:closediffinkspace} it is enough to show that 
    \[A \subseteq X \text{ is closed} \iff \text{for all compact sets } C \subseteq X \text{ the intersection } A \cap C \text{ is closed in } C.\]
    The forward direction is trivial.
    Let $A$ be a set such that $A \cap C$ is closed in $C$ for every compact set $C$. 
    Since $X$ is a sequential space it is enough to show that $A$ is sequentially closed or by the previous lemma $A  = \seqclosure{A}$.
    The inclusion $A \subseteq \seqclosure{A}$ is obvious. 
    For the backward inclusion take $x \in \seqclosure{A}$. 
    We need to show that $x \in A$.
    By definition there is a sequence $(a_n)_{n \in \bN} \subseteq A$ that converges to $x$. 
    It is well known (and can be shown directly from the definition of compactness) that the set $\{a_n \mid n \in \bN\} \cup {x}$ is compact as the set of terms of a sequence together with the limit point of that sequence.
    By assumption that gives us that $A \cap (\{a_n \mid n \in \bN\} \cup {x})$ is closed in $\{a_n \mid n \in \bN\} \cup {x}$. 
    In other words there is a closed set $B$ such that 
    \[A \cap (\{a_n \mid n \in \bN\} \cup {x}) = B \cap (\{a_n \mid n \in \bN\} \cup {x}).\]
    With that we get
    \[x \in A \iff x \in A \cap (\{a_n \mid n \in \bN\} \cup {x}) = B \cap (\{a_n \mid n \in \bN\} \cup {x}) \iff x \in B\]
    and for all $n \in \bN$ we get $a_n \in B$ in the exact same way. 
    Thus $(a_n)_{n \in \bN} \subseteq B$.
    Since $B$ is in particular sequentially closed this gives us $x \in B$ which is enough by the above equivalence.
\end{proof}

In particular sequential spaces include metric spaces: 

\begin{lem}
    Metric spaces are sequential spaces.
\end{lem}
\begin{proof}
    Let $X$ be a metric space and $A$ be a sequentially closed set. 
    We need to show that that $A$ is closed which is equivalent to $\compl{A}$ being open.
    Assume towards a contradiction that $\compl{A}$ is not open. 
    Then there is a point $x \in \compl{A}$ such that for every $n \in \bN$ the open ball $\oball{1/n}{x}$ contains a point $x_n \in A$. 
    But then we have a sequence $(x_n)_{n \in \bN} \subseteq A$ that converges to $x \in \compl{A}$. Thus $A$ is not sequentially closed. Contradiction. 
\end{proof}

\begin{cor}
    Metric spaces are k-spaces.
\end{cor}

Lastly we will discuss spaces that are not k-spaces:

\begin{lem}
    Let $X$ be an anti-compact T$_1$ space.
    Then $X_c$ has discrete topology.
\end{lem}
\begin{proof}
    Let $A \subseteq X_c$ be any set. We need to show that it is open. 
    By the definition of the k-ification it is enough to show that $A \cap C$
    is open in $C$ for every compact set $C \subseteq X$. 
    Since $X$ is anti-compact $C$ is finite.
    And by T$_1$ every finite set has discrete topology. 
    Thus $A \cap C$ is open in $C$ and $X_c$ has discrete topology.
\end{proof}

\begin{cor}
    Let $X$ be a non-discrete anti-compact T$_1$ space.
    Then $X$ is not a k-space.
\end{cor}
\begin{proof}
    This follows easily from the previous lemma and lemma \ref{lem:kificationkspace}.
\end{proof}

That leads us to our first concrete example of a space that is not a k-space: 

\begin{example}
    Let $X$ be any uncountable set.
    Equip $X$ with the cocountable topology, i.e. let a set $A \subseteq X$ be open iff $A = \varnothing$ or $\compl{A}$ is countable.
    Then $X$ is not a k-space.
\end{example}
\begin{proof}
    It is easy to see by going through the axioms that the cocountable topology is indeed a topology. 
    We will now show that this space satisfies the conditions of the previous corollary.
    $X$ is clearly non-discrete. 
    To see that $X$ is a T$_1$ space take two distinct points $a$ and $b$. 
    Now let $A$ be the set $X\setminus \{b\}$. 
    This set is open since $\{b\}$ is countable and it obviously does not contain $b$. 
    We lastly need to show that that $X$ is anti-compact. 
    To do that take any set $A \subseteq X$.
    Pick an (if possible infinite) countable subset $B \subseteq A$. 
    Now for every $b \in B$ define $U_b = (X \setminus B) \cup \{b\}$.
    Since $\compl{U_b} = B \setminus\{b\}$ is countable $U_b$ is open for every $b \in B$.
    It is also easy to see that $A \subseteq \bigcup_{b \in B}U_b$. 
    Thus $(U_b)_{b \in B}$ is an open cover of $A$. 
    But since for every $b \in B$ there is no $b' \in B$ with $b \ne b'$ and $b \in U_{b'}$, $(U_b)_{b \in B}$ cannot have a proper subcover.
    Therefore $A$ can only be compact if all these possible covers are already finite. 
    That can only be the case if $B$ and with that $A$ are finite. 
\end{proof}

Other examples can be found on $\pi$-base \cite{PiBase2024}.
