\subsection{Subcomplexes}

One important way to get a new CW-complex from an existing one is to consider subcomplexes which we will discuss in this section. 

Let $X$ be a CW-complex. A subcomplex of $X$ is defined as follows:

\begin{defi} \label{defi:subcomplex}
    A subcomplex of $X$ is a set $E \subseteq X$ together with a set $J_n \subseteq I_n$ for every $n \in \bN$ such that:
    \begin{enumerate}
        \item $E$ is closed.
        \item $\bigcup_{n \in \bN} \bigcup_{i \in J_n} \openCell{n}{i} = E$.
    \end{enumerate}
\end{defi}

Note that here we want $E$ to be the union of the open cells instead of the union of the closed cells as in definition \ref{defi:CWComplex2}. 
But we can prove the other version easily: 

\begin{lem} \label{lem:subcomplexunionclosed}
    Let $E \subseteq X$ be a subcomplex. 
    Then $\bigcup_{n \in \bN} \bigcup_{i \in J_n} \closedCell{n}{i} = E$.
\end{lem}
\begin{proof}
    Let $n \in \bN$ and $i \in J_n$. 
    It is enough to show that $\closedCell{n}{i} \subseteq E$. 
    By lemma \ref{lem:closureopencell} $\closedCell{n}{i} = \closure{\openCell{n}{i}}$. 
    Since $E$ is closed by property (i) $\closedCell{n}{i} \subseteq E$ is equivalent to $\openCell{n}{i} \subseteq E$ which is true by property (ii).
\end{proof}

Here are some alternative ways to define subcomplexes. 
These are taken from chapter 7.4 in \cite{Jänich2001}.
The proof that these three notions are equivalent can be found in there. 
We will just show the direction that is useful to us. 

\begin{lem} \label{lem:subcomplex2}
    Let $E \subseteq X$ and $J_n \subseteq I_n$ for $n \in \bN$ be such that 
    \begin{enumerate}
        \item For every $n \in \bN$ and $i \in I_n$ we have $\closedCell{n}{i} \subseteq E$. 
        \item $\bigcup_{n \in \bN} \bigcup_{i \in J_n} \openCell{n}{i} = E$.
    \end{enumerate}
    Then $E$ is a subcomplex of $X$.
\end{lem}
\begin{proof}
    Property (ii) in definition \ref{defi:subcomplex} is clear immediately. 
    So we only need to show that $E$ is closed. 
    We apply lemma \ref{lem:closediffinteropenorclosed} which means we only need to show that for every $n \in \bN$ and $i \in I_n$ either $E \cap \closedCell{n}{i}$ or $E \cap \openCell{n}{i}$ is closed. 
    So let $n \in \bN$ and $i \in I_n$. 
    We differentiate the cases $i \in J_n$ and $i \notin J_n$.
    For the first one notice that by property (i) $E$ can be expressed as a union of closed cells: $E = \bigcup_{m \in \bN} \bigcup_{j \in J_n} \openCell{m}{j} \subseteq \bigcup_{m \in \bN} \bigcup_{j \in J_n} \closedCell{m}{j} \subseteq E$. 
    This gives us $E \cap \closedCell{n}{i} = \closedCell{n}{i}$ which is closed by lemma \ref{lem:closedCellclosed}. 
    Now for the case $i \notin J_n$ the disjointness of the open cells of $X$ gives us that $E \cap \openCell{n}{i} = (\bigcup_{m \in \bN} \bigcup_{j \in J_n} \openCell{m}{j}) \cap \openCell{n}{i} = \varnothing$ which is obviously closed. 
\end{proof}

And here is a third way to express the property of being a subcomplex:

\begin{lem}
    Let $E \subseteq X$ and $J_n \subseteq I_n$ for $n \in \bN$ be such that 
    \begin{enumerate}
        \item $E$ is a CW-complex with respect to the cells determined by $X$ and $J_n$.
        \item $\bigcup_{n \in \bN} \bigcup_{i \in J_n} \openCell{n}{i} = E$.
    \end{enumerate}
    Then $E$ is a subcomplex of $X$.
\end{lem}
\begin{proof}
    We will show that this satisfies the properties of the construction above in lemma \ref{lem:subcomplex2}.
    Property (ii) is again immediate. 
    Property (i) combined with the definition \ref{defi:CWComplex2} of a CW-complex immediately gives us property (i) of lemma \ref{lem:subcomplex2}.
\end{proof}

Now we can show that a subcomplex is indeed again a CW-complex: 

\begin{lem}
    Let $E \subseteq X$ together with $J_n \subseteq I_n$ for every $n \in \bN$ be a subcomplex of the CW-complex $X$. 
    Then $E$ is again a CW-complex with respect to the cells determined by $J_n$ and $X$.
\end{lem}
\begin{proof}
    We show this by verifying the properties in the definition \ref{defi:CWComplex2} of a CW-complex. 
    Properties (i) and (ii) are immediate and we already covered property (v) in lemma \ref{lem:subcomplexunionclosed}.

    Let us consider property (iii) i.e. closure finiteness. 
    So let $n \in \bN$ and $i \in J_n$. 
    By closure finiteness of $X$ we know that there is a finite set $E \subseteq \bigcup_{m < n} I_n$ such that $\cellFrontier{n}{i} \subseteq \bigcup_{e \in E}e$. 
    We define $E' \coloneq \{\openCell{m}{j} \in E \mid j \in J_m\}$. 
    We want to show that $\cellFrontier{n}{i} \subseteq \bigcup_{e \in E'}e$. 
    Take $x \in \cellFrontier{n}{i}$. 
    By $\cellFrontier{n}{i} \subseteq \bigcup_{e \in E}e$ there is an $\openCell{m}{j} \in E$ such that $x \in \openCell{m}{j}$. 
    It is obviously enough to show that $j \in J_m$. 
    By lemma \ref{lem:subcomplexunionclosed} we know that $x \in \cellFrontier{n}{i} \subseteq \closedCell{n}{i} \subseteq E$.
    But since $E = \bigcup_{m' \in \bN} \bigcup_{j' \in J_{m'}} \openCell{m'}{j'}$ there is $m' \in \bN$ and $j' \in J_{m'}$ such that $x \in \openCell{m'}{j'}$. 
    We know that the open cells of $X$ are disjoint which gives us $(m, j) = (m', j')$. 
    That directly implies $j \in J_m$ which we wanted to show. 

    Lastly we need to show property (iv), i.e. that $E$ has weak topology. 
    Like in a lot of our other proofs $A \subseteq E$ being closed implies that $A \cap \closedCell{n}{i}$ is closed for every $n \in \bN$ and $i \in J_n$. 
    So now take $A \subseteq E$ such that $A \cap \closedCell{n}{i}$ is closed in $E$ for every $n \in \bN$ and $i \in J_n$. 
    We need to show that $A$ is closed in $E$.
    It is enough to show that $A$ is closed in $X$. 
    We apply lemma \ref{lem:closediffinteropenorclosed} which means we only need to show that for every $n \in \bN$ and $j \in I_n$ either $A \cap \closedCell{n}{j}$ or $A \cap \openCell{n}{j}$ is closed. 
    We look at two cases. 
    Firstly consider $j \in J_n$. 
    Then $A \cap \closedCell{n}{i}$ is closed in $E$ by assumption.
    By the definition of the subspace topology this means that there exists a closed set $B \subseteq X$ such that $A \cap \closedCell{n}{i} = E \cap B$. 
    But since $E$ is closed by assumption (i) of definition \ref{defi:subcomplex} of a subcomplex that means that $A \cap \closedCell{n}{i}$ is the intersection of two closed sets in $X$ making it also closed. 
    Now let us cover the case $j \notin J_n$. 
    This gives us $A \cap \openCell{n}{j} \subseteq E \cap \openCell{n}{j} = (\bigcup_{m \in \bN} \bigcup_{i \in J_m} \openCell{m}{i}) \cap \openCell{n}{j} = \varnothing$ where the last equality holds since the open cells of $X$ are pairwise disjoint. 
    Thus $A \cap \openCell{n}{j} = \varnothing$ which is obviously closed.
\end{proof}

Now let us look at some properties of subcomplexes: 

\begin{lem}
    A union of subcomplexes $(E_i)_{i \in \iota}$ of $X$  with indexing sets $(I_{i, n})_{i \in \iota, n \in \bN}$ is again a subcomplex of $X$ with the indexing set $\bigcup_{i \in \iota} I_{i, n}$ for every $n \in \bN$.
\end{lem}
\begin{proof}
    We show that this construction satisfies the assumptions of lemma \ref{lem:subcomplex2}. 
    Property (ii) follows easily from that the fact that each of the subcomplexes $E_i$ is the union of its open cells. 
    So let us look at property (i).
    Take $n \in \bN$ and $j \in \bigcup_{i \in \iota} I_{i, n}$. 
    Then there is a $i \in \iota$ such that $j \in I_{i, n}$. 
    With lemma \ref{lem:subcomplexunionclosed} we get 
    $\closedCell{n}{j} \subseteq \bigcup_{n \in \bN}\bigcup_{j \in I_{i, n}} \closedCell{n}{j} = E_i \subseteq \bigcup_{i \in \iota} E_i$ which means we are done.
\end{proof}

\begin{rem}
    It is easy to see that taking a finite union of finite subcomplexes of $X$ yields again a finite subcomplex of $X$.
\end{rem}