\section{Definition and basic properties of a CW-complexes}

The modern definition of a CW-complex is the following:

\begin{defi}\label{defi:CWComplex1}
    Let $X$ be a topological space.
    A \emph{CW-complex} on $X$ is a filtration
    $X_0 \subseteq X_1 \subseteq X_2 \subseteq \dots$ such that
    \begin{enumerate}
        \item For every $n \ge 0$ there is a pushout of topological spaces
            \[
            \begin{tikzcd}[row sep = 1.5cm, column sep = 1.5cm]
                \displaystyle \coprod_{i \in I_n}S_i^{n - 1} \arrow[r, "\coprod_{i \in I_n"}q_i^n"] \arrow[d, hook, "\coprod_{i \in I_n} j_i"'] & X_{n- 1} \arrow[d] \\
                \displaystyle \coprod_{i \in I_n}D_i^n \arrow[r, "\coprod_{i \in I_n} Q_i^n"'] & X_n
            \end{tikzcd}
            \]
            where $I_n$ is any indexing set and $j_i \colon S_i^{n - 1} \to D_i^n$ is the usual inclusion for every $i \in I_n$.
        \item We have $X = \bigcup_{n \ge 0} X_n$.
        \item $X$ has weak topology, i.e. $A \subseteq X$ is open $\iff A \cap X_n$ is open in $X_n$ for every $n$.
    \end{enumerate}
    $X_n$ is called the \emph{$n$-skeleton}.
    An element $e^n \in \pi_0 (X_n \setminus X_{n - 1})$ is called an \emph{(open) $n$-cell}.
    $Q_i^n$ is called a \emph{characteristic map}.
\end{defi}

\symbolindex{$D^n$}{The closed unit disk in $\bR^n$, i.e. $D^n \coloneq \{x \in \bR^n \mid \norm{x} \le 1\}$.}{}
\symbolindex{$S^n$}{The boundary of the unit disk in $\bR^n$, i.e. $S^n \coloneq \{x \in \bR^n \mid \norm{x} = 1\}$.}{}

In this thesis we will however focus on the historical definition of CW-complexes first presented by \Citeauthor{Whitehead2018} which can be found in \cite{Whitehead2018}.

\begin{defi}\label{defi:CWComplex2}
    Let $X$ be a Hausdorff space.
    A \emph{CW-complex} on $X$ consists of a family of indexing sets $(I_n)_{n \in \bN}$ and a family of maps $(Q_i^n\colon D_i^n\rightarrow X)_{n \ge 0, i \in I_n}$ s.t.
    \begin{enumerate}
        \item $\restrict{Q_i^n}{\interior{D_i^n}}\colon \interior{D_i^n} \rightarrow Q_i^n(\interior{D_i^n})$ is a homeomorphism. We call $\openCell{n}{i} \coloneq Q_i^n(\interior{D_i^n})$ an \emph{(open) $n$-cell} (or a cell of dimension $n$) and $\closedCell{n}{i} \coloneq Q_i^n(D_i^n)$ a \emph{closed $n$-cell}.
        \item For all $n, m \in \bN$, $i \in I_n$ and $j \in I_m$ where $(n, i) \ne (m, j)$ the cells $\openCell{n}{i}$ and $\openCell{m}{j}$ are disjoint.
        \item For each $n \in \bN$, $i \in I_n$, $Q_i^n(\boundary D_i^n)$ is contained in the union of a finite number of closed cells of dimension less than $n$.
        \item $A \subseteq X$ is closed iff $Q_i^n(D_i^n) \cap A$ is closed for all $n \in \bN$ and $i \in I_n$.
        \item $\bigcup_{n \ge 0}\bigcup_{i \in I_n} Q_i^n(D_i^n) = X$.
    \end{enumerate}
    We call $Q_i^n$ a \emph{characteristic map} and $\cellFrontier{n}{i} \coloneq Q_i^n(\boundary D_i^n)$ the \emph{frontier of the $n$-cell} for any $i$ and $n$.
    Additionally we define $X_n \coloneq \bigcup_{m < n + 1} \bigcup_{i \in I_m} \closedCell{m}{i}$ and call it the \emph{$n$-skeleton} of $X$ for $-1 \le n \le \infty$.
\end{defi}

\symbolindex{$\openCell{n}{}$}{An (open) $n$-cell, i.e. $\openCell{n}{} \coloneq Q^n(\interior{D^n})$. See definition \ref{defi:CWComplex2}.}{}
\symbolindex{$\closedCell{n}{}$}{A closed $n$-cell, i.e. $\closedCell{n}{} \coloneq Q^n(D^n)$. See definition \ref{defi:CWComplex2}.}{}
\symbolindex{$\cellFrontier{n}{}$}{The frontier of an $n$-cell, i.e. $\cellFrontier{n}{} \coloneq Q^n(\boundary D^n)$. See definition \ref{defi:CWComplex2}.}{}

For the rest of the section let $X$ be a CW-complex.

\begin{rem}
    Property (iii) in the above definition is called \emph{closure finiteness}.
    Property (iv) is called \emph{weak topology}.
    Whitehead named CW-complexes or long \emph{closure finite complexes with weak topology} after these two properties \cite{Whitehead2018}.
\end{rem}

Let us first answer the obvious question about the two definitions:

\begin{prop}
    Definition \ref{defi:CWComplex1} and \ref{defi:CWComplex2} are equivalent.
\end{prop}

The proof to this proposition is long, tedious and not relevant to this thesis so we will skip it here.
It can be found as the proof of Proposition A.2. in \cite{Hatcher2001}.
From here on the term CW-Complex will always refer to the older definition \ref{defi:CWComplex2}.
As such keep in mind that throughout this thesis any CW-complex will by definition be assumed to be Hausdorff.

\begin{rem}
    The name \emph{open $n$-cell} and the notation $\cellFrontier{n}{i}$ can be confusing as an open $n$-cell is not necessarily open and $\cellFrontier{n}{i}$  is not necessarily the boundary of $\closedCell{n}{i}$.
\end{rem}

But at least the notion of a closed $n$-cell makes sense:

\begin{lem}\label{lem:closedCellclosed}
    $\closedCell{n}{i}$ is compact and closed for every $n \in \bN$ and $i \in I_n$.
    Similarly $\cellFrontier{n}{i}$ is compact and closed for every $n \in \bN$ and $i \in I_n$.
\end{lem}
\begin{proof}
    $D_i^n$ is compact.
    Therefore its image $Q_i^n(D_i^n) = \closedCell{n}{i}$ is compact as well.
    In a Hausdorff space any compact set is closed.
    Thus $\closedCell{n}{i}$ is closed.
    The proof for $\cellFrontier{n}{i}$ works in the same way.
\end{proof}

And luckily the following is also true:

\begin{lem}\label{lem:closureopencell}
    $\closure{\openCell{n}{i}} = \closedCell{n}{i}$ for every $n \in \bN$ and $i \in I_n$.
\end{lem}
\begin{proof}
    Since $\openCell{n}{i} \subseteq \closedCell{n}{i}$ and $\closedCell{n}{i}$ is closed by the lemma above, the left inclusion is trivial.
    So let us show now that $\closedCell{n}{i} \subseteq \closure{\openCell{n}{i}}$.
    This statement can be rewritten as $Q_i^n \left ( \closure{D_i^n} \right ) \subseteq \closure{Q_i^n(D_i^n)}$.
    It is generally true for any continuous map that the closure of the image is contained in the image of the closure.
    Thus we are done.
\end{proof}

Now let us define what it means for a CW-complex to be finite:

\begin{defi}
    Let $X$ be a CW-complex.
    We call $X$ \emph{of finite type} if there are only finitely many cells in each dimension, i.e. if $I_n$ is finite for all $n \in \bN$.
    $X$ is said to be \emph{finite dimensional} if there is an $n \in \bN$ such that $X = X_n$.
    Finally, $X$ is called \emph{finite} if it is of finite type and finite dimensional.
\end{defi}

If we already know that the CW-complex we want to construct will be finite or of finite type we can relax some of the conditions:

\begin{rem}~
    \begin{enumerate}
        \item For a CW-complex of finite type condition (iii) in definition \ref{defi:CWComplex2} follows from the following:
        For each $n \in \bN$, $i \in I_n$ $Q_i^n(\boundary D_i^n)$ is contained in $\bigcup_{m \le n - 1}\bigcup_{i \in I_m}\openCell{m}{i}$.
        \item Additionally for a finite CW-complex condition (iv) in definition \ref{defi:CWComplex2} is follows from the other conditions.
    \end{enumerate}
\end{rem}
\begin{proof}
    Let us begin with statement (i).
    Take $n \in \bN$ and $i \in I_n$.
    We need to show that $Q_i^n(\boundary D_i^n)$ is contained in a finite number of cells of a lower dimension.
    But by assumption we have $Q_i^n(\boundary D_i^n) \subseteq \bigcup_{m \le n - 1}\bigcup_{i \in I_m}\openCell{m}{i}$ which in this case is made up of finitely many cells.
    Now we can move on to statement (ii).
    We need to prove condition (iv) of definition \ref{defi:CWComplex2}, i.e.
    \[A \subseteq X \text{ is closed} \iff \closedCell{n}{i} \cap A \text{ is closed for all } n \in \bN \text{ and } i \in I_n.\]
    For the forward direction notice that $\closedCell{n}{i} \cap A$ is just the intersection of two closed sets by assumption and lemma \ref{lem:closedCellclosed}.
    As such it is closed.
    For the backward direction take an $A \subseteq X$ such that $\closedCell{n}{i}$ is closed for all $n \in \bN$ and $i \in I_n$.
    We need to show that $A$ is closed.
    But using condition (v) of definition \ref{defi:CWComplex2} we get
    \[A = A \cap \bigcup_{n \ge 0}\bigcup_{i \in I_n} \closedCell{n}{i} = \bigcup_{n \ge 0}\bigcup_{i \in I_n} (A \cap \closedCell{n}{i})\]
    which by assumption is a finite union of closed sets, making $A$ closed.
\end{proof}

We can also think about the $n$-skeletons as being made up of open cells:

\begin{lem}\label{lem:skeletonunionopenCell}
    $X_n = \bigcup_{m < n + 1}\bigcup_{i \in I_m} \openCell{m}{i}$ for every $-1 \le n \le \infty$.
\end{lem}
\begin{proof}
    We show this by induction over $-1 \le n \le \infty$.
    For the base case assume that $n = -1$.
    Then we get $X_n = \bigcup_{m < 0}\bigcup_{i \in I_m}\closedCell{m}{i} = \varnothing = \bigcup_{m < 0}\bigcup_{i \in I_m} \openCell{m}{i}$.

    For the induction step assume that that the statement is true for $n$.
    We now show that it also holds for $n + 1$.
    \begin{equation*}
        \begin{split}
            X_{n + 1} &= \bigcup_{m < n + 2}\bigcup_{i \in I_m} \closedCell{m}{i} \\
            &=\bigcup_{i \in I_{n + 1}}\closedCell{n+1}{i} \cup \bigcup_{m < n + 1}\bigcup_{i \in I_m} \closedCell{m}{i} \\
            &= \bigcup_{i \in I_{n + 1}}\closedCell{n+1}{i} \cup X_n \\
            &\stackrel{(1)}{=} \bigcup_{i \in I_{n + 1}}\closedCell{n+1}{i} \cup \bigcup_{m < n + 1}\bigcup_{i \in I_m} \openCell{m}{i} \\
            &= \bigcup_{i \in I_{n + 1}}\openCell{n+1}{i} \cup \bigcup_{i \in I_{n + 1}}\cellFrontier{n+1}{i} \cup \bigcup_{m < n + 1}\bigcup_{i \in I_m} \openCell{m}{i} \\
            &\stackrel{(2)}{=} \bigcup_{i \in I_{n + 1}}\openCell{n+1}{i} \cup \bigcup_{m < n + 1}\bigcup_{i \in I_m} \openCell{m}{i} \\
            &= \bigcup_{m < n + 2}\bigcup_{i \in I_m} \openCell{m}{i}
        \end{split}
    \end{equation*}
    Where (1) holds by induction and (2) holds by closure finiteness (property (iii) in definition \ref{defi:CWComplex2}).

    Now we can move on to the case $n = \infty$.
    \begin{equation*}
        \begin{split}
            X_{\infty} &= \bigcup_{m < \infty + 1}\bigcup_{i \in I_m}\closedCell{m}{i} \\
            &= \bigcup_{m < \infty + 1}\bigcup_{l < m + 1}\bigcup_{i \in I_l}\closedCell{l}{i} \\
            &= \bigcup_{m < \infty + 1} X_m \\
            &\stackrel{(1)}{=} \bigcup_{m < \infty + 1}\bigcup_{l < m + 1}\bigcup_{i \in I_l}\openCell{l}{i} \\
            &= \bigcup_{m < \infty + 1}\bigcup_{i \in I_m}\closedCell{m}{i}
        \end{split}
    \end{equation*}
    Where (1) holds by induction.
\end{proof}

This also enables us to write $X$ as a union of open cells:

\begin{cor} \label{cor:complexeqopencells}
    $\bigcup_{n \ge 0}\bigcup_{i \in I_n} \openCell{n}{i} = X$.
\end{cor}

When we want to show that a set $A \subseteq X$ is closed the weak topology (property (iv) in \ref{defi:CWComplex2}) lets us reduce that question to an individual cell.
It is then often convenient to do strong induction over the dimension of the cell.
We now want to prove a lemma that makes this repeated process easier.
We first need the following:

\begin{lem}\label{lem:inductioncellFrontierclosed}
    Let $A \subseteq X$ be a set and $n$ a natural number.
    Assume that for every $m \le n$ and $j \in I_m$ the intersection $A \cap \closedCell{m}{j}$ is closed.
    Then $A \cap \cellFrontier{n + 1}{j}$ is closed for every $j \in I_{n + 1}$.
\end{lem}
\begin{proof}
    By closure finiteness of $X$ (property (iii) in \ref{defi:CWComplex2}) there is a set $E$ of cells of dimension lower than $n + 1$ such that $\cellFrontier{n + 1}{j} \subseteq \bigcup_{e \in E}\closure{e}$.
    This gives us
    \[A \cap \cellFrontier{n + 1}{j} = A \cap \bigcup_{e \in E}\closure{e} \cap \cellFrontier{n + 1}{j} = \bigcup_{e \in E}(A \cap \closure{e}) \cap \cellFrontier{n + 1}{j}.\]
    $\bigcup_{e \in E}(A \cap \closure{e})$ is closed as a finite union of sets that are by assumption closed and $\cellFrontier{n + 1}{j}$ is closed by lemma \ref{lem:closedCellclosed}. Therefore the intersection is also closed.
\end{proof}

Now we can move on to the lemma that we actually want.
We can think of this lemma as being a weaker condition than the weak topology i.e. property (iv) in \ref{defi:CWComplex2}.

\begin{lem}\label{lem:closediffinteropenorclosed}
    Let $A \subseteq X$ be a set such that for every $n > 0$ and $j \in I_n$ either $A \cap \openCell{n}{j}$ or $A \cap \closedCell{n}{j}$ is closed.
    Then $A$ is closed.
\end{lem}
\begin{proof}
    Since $X$ has weak topology, it is enough to show that $A \cap \closedCell{n}{j}$ is closed for every $n \in \bN$ and $i \in I_n$.
    We show this by doing strong induction over $n$.
    For the base case $n = 0$ notice that $\closedCell{0}{j}$ is a singleton and the intersection with a singleton is either that singleton or empty.
    As such the intersection is closed in both cases.

    Now let us move on to the induction step.
    Assume that for every $m \le n$ the statement already holds.
    We now need to show it for $n + 1$.
    By assumption either $A \cap \openCell{n + 1}{j}$ or $A \cap \closedCell{n + 1}{j}$ is closed. The second case is just immediately what we wanted to show.

    In the first case we can use that $A \cap \closedCell{n + 1}{j} = (A \cap \cellFrontier{n + 1}{j}) \cup (A \cap \openCell{n + 1}{j})$.
    The left part of the union is closed by lemma \ref{lem:inductioncellFrontierclosed} applied to the induction hypothesis.
    The right part of the union is closed by the assumption of our case.
    The union is therefore also closed.
\end{proof}

We can use the lemma we just proved to show that the $n$-skeletons are closed:

\begin{lem}\label{lem:levelclosed}
    $X_n$ is closed for every $n \in \bN$.
\end{lem}
\begin{proof}
    By the previous lemma \ref{lem:closediffinteropenorclosed} it is enough to show that for every $m \in \bN$ and $j \in I_m$ either $X_n \cap \openCell{m}{j}$ or $X_n \cap \closedCell{m}{j}$ is closed.
    We differentiate two cases bases on whether $m < n + 1$.
    First assume that $m < n + 1$ holds.
    Then by the definition of $X_n$ we get $X_n \cap \closedCell{m}{j} = (\bigcup_{m < n + 1} \bigcup_{i \in I_m} \closedCell{m}{i}) \cap \closedCell{m}{j} = \closedCell{m}{j}$ which is closed by lemma \ref{lem:closedCellclosed}.
    Now assume that it does not hold.
    Then by lemma \ref{lem:skeletonunionopenCell} we get $X_n \cap \openCell{m}{j} = (\bigcup_{m < n + 1} \bigcup_{i \in I_m} \openCell{m}{i}) \cap \openCell{m}{j} = \varnothing$ where the last equality holds because the open cells are pairwise disjoint by property (ii) in definition \ref{defi:CWComplex2}.
    The empty set is obviously closed.
\end{proof}

Another fact that can be quite helpful is a version of closure finiteness using open cells:

\begin{lem}
    For each $n \in \bN$ and $i \in I_n$ $\cellFrontier{n}{i}$ is contained in the union of a finite number of open cells of dimension less than $n$.
\end{lem}
\begin{proof}
    We show this by doing strong induction on $n$.
    For the base case $n = 0$ notice that $\cellFrontier{0}{i}$ is empty.

    Moving on to the induction step assume that the statement holds for all $m \le n$.
    We need to show that it also holds for $n + 1$.
    By closure finiteness there is a finite set $E$ of cells of dimension less than $n + 1$ such that $\cellFrontier{n + 1}{i} \subseteq \bigcup_{e \in E}\closure{e}$.
    If we can show that for every $e \in E$ there is a finite set $E_e$ of cells of dimension less than $n + 1$ such that $\closure{e} \subseteq \bigcup_{e' \in E}e'$, we would then be done since $\cellFrontier{n + 1}{i} \subseteq \bigcup_{e \in E}\closure{e} \subseteq \bigcup_{e \in E}\bigcup_{e' \in E}e'$.

    So take $e \in E$.
    By the induction hypothesis there is a finite set $E'_e$ of cells of a lesser dimension than that of $e$ such that $\boundary e \subseteq \bigcup_{e' \in E'_e}e'$.
    This gives us $\closure e = \boundary e \cup e \subseteq (\bigcup_{e' \in E'_e}e') \cup e$ which finishes the proof.
\end{proof}

Let us now look at some more ways to show that sets in $X$ are closed. 

\begin{lem}\label{lem:closediffskeleton}
    $A \subseteq X$ is closed iff $A \cap X_n$ is closed for every $n \in \bN$.
\end{lem}
\begin{proof}
    The forward direction follows directly from lemma \ref{lem:levelclosed}. 
    For the backward direction take $A \subseteq X$ such that $A \cap X_n$ is closed for every $n$.
    Since $X$ has weak topology we need to show that $A \cap \closedCell{n}{i}$ is closed for every $n \in \bN$ and $i \in I_n$. 
    But $A \cap \closedCell{n}{i} = A \cap X_n \cap \closedCell{n}{i}$ which is closed by assumption and lemma \ref{lem:closedCellclosed}.
\end{proof}

When we use this lemma with induction we might want the following for the induction step:

\begin{lem}
    Let $A \subseteq X$. $A \cap X_{n + 1}$ is closed iff $A \cap X_n$ and $A \cap \closedCell{n + 1}{j}$ are closed for every $j \in I_{n + 1}$.
\end{lem}
\begin{proof}
    For the forward direction notice that $A \cap X_n = A \cap X_{n + 1} \cap X_n$ which is closed by assumption and lemma \ref{lem:levelclosed} and $A \cap \closedCell{n + 1}{j} = A \cap X_{n + 1} \cap \closedCell{n + 1}{j}$ which is closed by assumption and lemma \ref{lem:closedCellclosed}.
    For the backwards direction we apply lemma \ref{lem:closediffinteropenorclosed}. 
    We now need to show that for every $m \in \bN$ and $j \in I_m$ either $A \cap \openCell{m}{j}$ or $A \cap \closedCell{m}{j}$ is closed. 
    We differentiate three different cases. 
    First let us look at the case $m \le n$.
    Then $A \cap X_{n + 1} \cap \closedCell{m}{j} = A \cap \closedCell{m}{j} = A \cap X_n \cap \closedCell{m}{j}$ which is closed by assumption and lemma \ref{lem:closedCellclosed}. 
    No we consider $m = n + 1$. 
    Then $A \cap X_{n + 1} \cap \closedCell{n + 1}{j}$ = $A \cap \closedCell{n + 1}{j}$ which is closed by assumption. 
    Lastly we show the claim for $m > n + 1$. 
    Here we get $A \cap X_{n + 1} \cap \openCell{m}{j} = A \cap (\bigcup_{l < n + 1}\bigcup_{i \in I_l}\openCell{l}{j} ) \cap \openCell{m}{j} = \varnothing$ where we used lemma \ref{lem:skeletonunionopenCell} and the fact that different open cells are disjoint (property (ii) in definition \ref{defi:CWComplex2}). 
    The empty set is obviously closed.
\end{proof}

With that we can write a new strong induction principle for showing that sets in a CW-complex are closed: 

\begin{lem}\label{lem:inductionlevel}
    Let $A \subseteq X$ be a set such that for all $n \in \bN$ if for all $m \le n$ the intersection $A \cap X_m$ is closed then for all $j \in I_{n + 1}$ the intersection $A \cap \closedCell{n + 1}{j}$ is closed. 
    Then $A$ is closed.
\end{lem}
\begin{proof}
    By lemma \ref{lem:closediffskeleton} it is enough to show that for all $n \in \bN$ the set $A \cap X_n$ is closed. 
    We do strong induction over $n$ starting at $-1$. 
    For the base case notice that $X_{-1} = \varnothing$.
    Now for the induction step assume that $A \cap X_m$ is closed for all $m \le n$. 
    We need to show that $A \cap X_{n + 1}$ is closed as well. 
    By the previous lemma it is enough to show that $A \cap X_n$ and $A \cap \closedCell{n + 1}{j}$ are closed for all $j \in I_{n + 1}$. 
    But the first one is closed by induction hypothesis and the second one is closed by our assumption applied to the induction hypothesis.
\end{proof}

We can now use all these new techniques to show some important properties of CW-complexes: 

\begin{lem}
    $X_0$ is discrete.
\end{lem}
\begin{proof}
    We want to show that every set $A \subseteq X_0$ is closed in $X_0$.
    It is enough if $A$ is closed in $X$. 
    We apply lemma \ref{lem:closediffinteropenorclosed}. 
    Take $n > 0$ and $i \in I_n$. 
    We show that $A \cap \openCell{n}{i}$ is closed. 
    But using \ref{lem:skeletonunionopenCell} that different open cells are disjoint we have $A \cap \openCell{n}{i} = A \cap X_0 \cap \openCell{n}{i} = A \cap (\bigcup_{m < 1}\bigcup_{j \in I_m}\openCell{m}{j}) \cap \openCell{n}{i} = \varnothing$ which is closed.
\end{proof}

The proof of the following lemma is based on the proof of Proposition A.1. in \cite{Hatcher2001}.

\begin{lem} \label{lem:compactintersectsonlyfinite}
    For every compact set $C \subseteq X$ the set of all open cells $\openCell{n}{i}$ such that $\openCell{n}{i} \cap C \ne \varnothing$ is finite.
\end{lem}
\begin{proof}
    Assume towards a contradiction that the set $S \coloneq \{n \in \bN, i \in I_n \mid \openCell{n}{i} \cap C \ne \varnothing \}$ is infinite. 
    For every pair $(n, i) \in S$ pick a point $p_{n, i} \in \openCell{n}{i} \cap C$. 
    Since the open cells are pairwise disjoint we know that the set $P \coloneq \{p_{n, i} \mid (n, i) \in S\}$ is also infinite.
    We will now show that $P$ is discrete and compact. 
    Then $P$ must be finite which is a contradiction.
    For both compactness and discreteness we will need that every set $A \subseteq P$ is closed in $X$.

    So let $A \subseteq P$. 
    We apply lemma \ref{lem:inductionlevel}. 
    Assuming that for all $m \le n$ the intersection $A \cap X_m$ is closed, we need to show that $A \cap \closedCell{n + 1}{j}$ is closed for every $j \in I_{n + 1}$.
    Since $A \cap \closedCell{n + 1}{j} = (A \cap \cellFrontier{n + 1}{j}) \cup (A \cap \openCell{n + 1}{j})$ and $A \cap \cellFrontier{n + 1}{j} = A \cap X_{n} \cap \cellFrontier{n + 1}{j}$ is closed by lemma \ref{lem:closedCellclosed} and the assumption, it is enough to show that $A \cap \openCell{n + 1}{j}$ is closed. 
    If the intersection $A \cap \openCell{n + 1}{j}$ is empty then we are done. 
    So assume that there is an $x \in A \cap \openCell{n + 1}{j}$. 
    Since $x \in A \subseteq P$ there is $(m, i) \in S$ such that $p_{m, i} = x$. 
    But the open cells of $X$ are pairwise disjoint so it must be that $(m, i) = (n + 1, j)$ and therefore $p_{n + 1, j} = x$. 
    Thus $A \cap \openCell{n + 1}{j} = \{p_{n + 1, j}\}$ which is closed since every singleton in a Hausdorff space is closed. 

    This directly gives us that the subspace topology on $P$ is discrete. 
    For compactness notice that by what we just did $P$ is closed and as a closed subset of the compact set $C$ it is also compact. 
    This is a contradiction to the fact that $P$ is infinite as explained above. 
\end{proof}

This lemma helps us prove the following characterisation of finite CW-complexes: 

\begin{lem}
    $X$ is a finite CW-complex iff $X$ compact.
\end{lem}
\begin{proof}
    For the forward direction we know that $X = \bigcup_{n \in \bN}\bigcup_{i \in I_n} \closedCell{n}{i}$ which by assumption and lemma \ref{lem:closedCellclosed} is compact as a finite union of compact sets. 

    The backward direction follows from lemma \ref{lem:compactintersectsonlyfinite} and corollary \ref{cor:complexeqopencells}.
\end{proof}