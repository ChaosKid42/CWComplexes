\section{K-spaces and the k-ification}

Before we can move on to discuss the product of CW-complexes we need to discuss its topology.
Therefore we will study k-spaces and the k-ification in this section.

A k-space or also called a compactly generated space is defined for our purposes as follows. 
Note that we mean quasi-compactness when talking about compactness.

\begin{defi}
    Let $X$ be a topological space. 
    We call $X$ a k-space if 
    \[A \subseteq X \text{ is open} \iff \text{for all compact sets } C \subseteq X \text{ the intersection } A \cap C \text{ is open in } C.\]
\end{defi}

There are a lot of different definitions in the literature. 
The most popular ones all agree on Hausdorff spaces. 
An overview of these different notions can be found on Wikipedia \cite{Wikipedia2024}.

It will also be helpful to characterise closed sets in the same way as the open sets: 

\begin{lem}\label{lem:closediffinkspace}
    Let $X$ be a k-space. 
    Then 
    \[A \subseteq X \text{ is closed} \iff \text{for all compact sets } C \subseteq X \text{ the intersection } A \cap C \text{ is closed in } C.\]
\end{lem}
\begin{proof}
    The forward direction is trivial. 
    So let $A \subseteq X$ be a set such that for all compact sets $C \subseteq X$ the intersection $A \cap C$ is closed in $C$. 
    It is enough to show that $\compl{A}$ is open. 
    By definition of the k-space that is the case if for every compact set $C \subseteq X$ the intersection $\compl{A} \cap C$ is open in $C$. 
    Take any compact $C \subseteq X$.
    By assumption $A \cap C$ is closed in $C$.
    Since $A \cap C$ is the complement of $\compl{A} \cap C$ in $C$, this immediately gives us that $\compl{A} \cap C$ is open in $C$.
\end{proof}

We also define a way to make any topological space into a k-space which we call the k-ification: 

\begin{defi}
    Let $X$ be a topological space. 
    We can define another topological space $X_c$ on the same set by setting
    \[A \subseteq X_c \text{ is open} \iff \text{for all compact sets } C \subseteq X \text{ the intersection } A \cap C \text{ is open in } C.\]
    We call $X_c$ the k-ification of $X$.
\end{defi}

It is easy to see that this gives us a finer topology: 

\begin{lem}\label{lem:finertopology}
    $A \subseteq X$ is open $\implies A \subseteq X_c$ is open.
\end{lem}

Again it it useful to characterise the closed sets in the k-ification: 

\begin{lem}
    $A \subseteq X_c$ is closed $\iff A \cap C$ is closed in $C$ for all compact sets $C \subseteq X$.
\end{lem}
\begin{proof}
    Completely analogue to the proof of lemma \ref{lem:closediffinkspace}.
\end{proof}

To show that the k-ification actually fulfils its purpose of turning any space into a k-space, we first need the following lemma:

\begin{lem}\label{lem:compactiffcompact}
    $A \subseteq X$ is compact $\iff A \subseteq X_c$ is compact.
\end{lem}
\begin{proof}
    For the backward direction notice that lemma \ref{lem:finertopology} is another way of stating that the map $\id \colon X_c \to X$ is continuous. 
    As the image of a compact set under a continuous map, that makes $A \subseteq X$ compact. 

    For the forward direction take $A \subseteq X$ compact. 
    To show that $A \subseteq X_c$ is compact, take an open cover $(U_i)_{i \in \iota}$ of $A$ in $X_c$. 
    For every $i \in \iota$ there is by definition of the k-ification an open set $V_i \subseteq X$ such that $V_i \cap A = U_i \cap A$.  
    $(V_i)_{i \in \iota}$ is an (open) cover of $A$ in $X$: 
    \[A = A \cap \bigcup_{i \in \iota} U_i = \bigcup_{i \in \iota} (A \cap U_i) = \bigcup_{i \in \iota} (A \cap V_i) = A \cap \bigcup_{i \in \iota} V_i \subseteq \bigcup_{i \in \iota} V_i.\]
    Thus there is a finite subcover $(V_i)_{i \in \iota'}$ of $A$ in $X$.
    $(U_i)_{i \in \iota'}$ is now a finite subcover of $A$ in $X_c$: 
    \[A = A \cap \bigcup_{i \in \iota'} V_i = \bigcup_{i \in \iota'} (A \cap V_i) = \bigcup_{i \in \iota'} (A \cap U_i) = A \cap \bigcup_{i \in \iota'} U_i \subseteq \bigcup_{i \in \iota'} U_i.\]
\end{proof}

Now we are ready to move on to the promised lemma:

\begin{lem}
    $X_c$ is a k-space for every topological space $X$.
\end{lem}
\begin{proof}
    We need to show that a set $A \subseteq X_c$ is open iff $A \cap C$ is open in $C$ for every compact set $C \subseteq X_c$. The forward direction is again trivial. 
    
    For the backward direction take a set $A \subseteq X_c$ such that for every compact set $C \subseteq X_c$ the intersection $A \cap C$ is open in $C$. 
    By the definition of the k-ification it is enough to show that for every compact set $C \subseteq X$ the intersection $A \cap C$ is open in $C$. So let $C \subseteq X$ be a compact set. 
    By \ref{lem:compactiffcompact} $C$ is also compact in $X$.
    By assumption this means that $A \cap C$ is open in $C \subseteq X_c$ (in the subspace topology of the k-ification). Thus there is an open set $B \subset X_c$ such that $A \cap C = B \cap C$. 
    By the definition of the k-ification $B \cap C$ is open in $C \subseteq X$. 
    That means there is an open set $E \subseteq X$ such that $B \cap C = E \cap C$. 
    But that now gives us $A \cap C = B \cap C = E \cap C$ with which we can conclude that $A \cap C$ is open in $C \subseteq X$ (in the subspace topology of the original topology of X).
\end{proof}

If we already have a k-space, then the k-ification just maintains the topology of our space:

\begin{lem}\label{lem:kificationkspace}
    Let $X$ be a k-space.
    Then the topologies of $X$ and $X_c$ coincide.
\end{lem}
\begin{proof}
    Notice that the characterisation of open sets in $X$ and $X_c$ respectively agree in this setting.
\end{proof}

\begin{cor}
    The k-ification is idempotent.
\end{cor}

Now we will characterise continuous maps to and from the k-ification. 
Going from the k-ification is not a big issue: 

\begin{lem}
    Let $f \colon X \to Y$ be a continuous map of topological spaces.
    Then $f \colon X_c \to Y$ is continuous.
\end{lem}
\begin{proof}
    This follows easily from lemma \ref{lem:finertopology}.
\end{proof}

A more interesting question is when a map to the k-ification is continuous. The following lemma is the first step towards the answer:

\begin{lem}
    Let $X$ be a compact space and $f \colon X \to Y$ be a continuous map. 
    Then $f \colon X \to Y_c$ is continuous.
\end{lem}

\begin{lem}
    Let $X$ be an anti-compact T$_1$ space.
    Then $X_c$ has discrete topology.
\end{lem}
\begin{proof}
    Let $A \subseteq X_c$ be any set. We need to show that it is open. 
    By the definition of the k-ification it is enough to show that $A \cap C$
    is open in $C$ for every compact set $C \subseteq X$. 
    Since $X$ is anti-compact $C$ is finite.
    And by T$_1$ every finite set has discrete topology. 
    Thus $A \cap C$ is open in $C$ and $X_c$ has discrete topology.
\end{proof}

\begin{cor}
    Let $X$ be a non-discrete anti-compact T$_1$ space.
    Then $X$ is not a k-space.
\end{cor}
\begin{proof}
    This follows easily from the previous lemma and lemma \ref{lem:kificationkspace}.
\end{proof}