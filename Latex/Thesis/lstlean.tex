% Listing style definition for the Lean Theorem Prover.
% Defined by Jeremy Avigad, 2015, by modifying Assia Mahboubi's SSR style.
% Unicode replacements taken from Olivier Verdier's unixode.sty

\lstdefinelanguage{lean} {

% Anything between $ becomes LaTeX math mode
mathescape=false,
% Comments may or not include Latex commands
texcl=false,

% keywords, list taken from lean-syntax.el
morekeywords=[1]{
import, prelude, protected, private, noncomputable, definition, meta, renaming,
hiding, parameter, parameters, begin, constant, constants,
lemma, variable, variables, theory,
print, theorem, example,
open, as, export, override, axiom, axioms, inductive, with,
structure, record, universe, universes,
alias, help, precedence, reserve, declare_trace, add_key_equivalence,
match, infix, infixl, infixr, notation, postfix, prefix, instance,
eval, reduce, check, end, this,
using, using_well_founded, namespace, section,
attribute, local, set_option, extends, include, omit, class,
raw, replacing,
calc, have, show, suffices, by, in, at, let, forall, Pi, fun,
exists, if, dif, then, else, assume, obtain, from, register_simp_ext, unless, break, continue,
mutual, do, def, run_cmd, const,
partial, mut, where, macro, syntax, deriving,
return, try, catch, for, macro_rules, declare_syntax_cat, abbrev},

% Sorts
morekeywords=[2]{Sort, Type, Prop},

% tactics, list taken from lean-syntax.el
morekeywords=[3]{
assumption,
apply, intro, intros, allGoals,
generalize, clear, revert, done, exact,
refine, repeat, cases, rewrite, rw,
simp, simp_all, contradiction,
constructor, injection,
induction,
},

% modifiers, taken from lean-syntax.el
% note: 'otherkeywords' is needed because these use a different symbol.
% this command doesn't allow us to specify a number -- they are put with [1]
% otherkeywords={
% [persistent], [notation], [visible], [instance], [trans_instance],
% [class], [parsing-only], [coercion], [unfold_full], [constructor],
% [reducible], [irreducible], [semireducible], [quasireducible], [wf],
% [whnf], [multiple_instances], [none], [decl], [declaration],
% [relation], [symm], [subst], [refl], [trans], [simp], [congr], [unify],
% [backward], [forward], [no_pattern], [begin_end], [tactic], [abbreviation],
% [reducible], [unfold], [alias], [eqv], [intro], [intro!], [elim], [grinder],
% [localrefinfo], [recursor]
% },

% Various symbols
literate=
{α}{{\ensuremath{\mathrm{\upalpha}}}}1
{β}{{\ensuremath{\mathrm{\upbeta}}}}1
{γ}{{\ensuremath{\mathrm{\upgamma}}}}1
{δ}{{\ensuremath{\mathrm{\updelta}}}}1
{ε}{{\ensuremath{\mathrm{\varepsilon}}}}1
{ζ}{{\ensuremath{\mathrm{\zeta}}}}1
{η}{{\ensuremath{\mathrm{\eta}}}}1
{θ}{{\ensuremath{\mathrm{\theta}}}}1
{ι}{{\ensuremath{\mathrm{\iota}}}}1
{κ}{{\ensuremath{\mathrm{\kappa}}}}1
{μ}{{\ensuremath{\mathrm{\mu}}}}1
{ν}{{\ensuremath{\mathrm{\nu}}}}1
{ξ}{{\ensuremath{\mathrm{\xi}}}}1
{π}{{\ensuremath{\mathrm{\mathnormal{\pi}}}}}1
{ρ}{{\ensuremath{\mathrm{\rho}}}}1
{σ}{{\ensuremath{\mathrm{\sigma}}}}1
{τ}{{\ensuremath{\mathrm{\tau}}}}1
{φ}{{\ensuremath{\mathrm{\varphi}}}}1
{χ}{{\ensuremath{\mathrm{\chi}}}}1
{ψ}{{\ensuremath{\mathrm{\psi}}}}1
{ω}{{\ensuremath{\mathrm{\omega}}}}1
{ϖ}{{\ensuremath{\mathrm{\mathnormal{\varpi}}}}}1 % KMB added for perfectoid stuff

{Γ}{{\ensuremath{\mathrm{\Gamma}}}}1
{Δ}{{\ensuremath{\mathrm{\Delta}}}}1
{Θ}{{\ensuremath{\mathrm{\Theta}}}}1
{Λ}{{\ensuremath{\mathrm{\Lambda}}}}1
{Σ}{{\ensuremath{\mathrm{\Sigma}}}}1
{Φ}{{\ensuremath{\mathrm{\Phi}}}}1
{Ξ}{{\ensuremath{\mathrm{\Xi}}}}1
{Ψ}{{\ensuremath{\mathrm{\Psi}}}}1
{Ω}{{\ensuremath{\mathrm{\Omega}}}}1

{ℵ}{{\ensuremath{\aleph}}}1

{≤}{{\color{symbolcolor}\ensuremath{\leq}}}1
{≥}{{\color{symbolcolor}\ensuremath{\geq}}}1
{≠}{{\color{symbolcolor}\ensuremath{\neq}}}1
{≈}{{\color{symbolcolor}\ensuremath{\approx}}}1
{≡}{{\color{symbolcolor}\ensuremath{\equiv}}}1
{≃}{{\color{symbolcolor}\ensuremath{\simeq}}}1

{∂}{{\color{symbolcolor}\ensuremath{\partial}}}1
{∆}{{\color{symbolcolor}\ensuremath{\triangle}}}1 % or \laplace?

{∫}{{\color{symbolcolor}\ensuremath{\int}}}1
{∑}{{\color{symbolcolor}\ensuremath{\mathrm{\Sigma}}}}1
%{Π}{{\ensuremath{\mathrm{\Pi}}}}1

{⊥}{{\color{symbolcolor}\ensuremath{\perp}}}1
{⊤}{{\color{symbolcolor}\ensuremath{\top}}}1
{∞}{{\color{symbolcolor}\ensuremath{\infty}}}1
{∂}{{\color{symbolcolor}\ensuremath{\partial}}}1

{∓}{{\color{symbolcolor}\ensuremath{\mp}}}1
{±}{{\color{symbolcolor}\ensuremath{\pm}}}1
{×}{{\color{symbolcolor}\ensuremath{\times}}}1

{⊕}{{\color{symbolcolor}\ensuremath{\oplus}}}1
{⊗}{{\color{symbolcolor}\ensuremath{\otimes}}}1
{⊞}{{\color{symbolcolor}\ensuremath{\boxplus}}}1

{∇}{{\color{symbolcolor}\ensuremath{\nabla}}}1
{√}{{\color{symbolcolor}\ensuremath{\sqrt}}}1

{⬝}{{\color{symbolcolor}\ensuremath{\cdot}}}1
{•}{{\color{symbolcolor}\ensuremath{\cdot}}}1
{∘}{{\color{symbolcolor}\ensuremath{\circ}}}1
{ᵒ}{{\color{symbolcolor}\ensuremath{^\circ}}}1
{`}{{\ensuremath{{}^\backprime}}}1
{'}{{\ensuremath{{}^\prime}}}1

%{⁻}{{\ensuremath{^{\textup{\kern1pt\rule{2pt}{0.3pt}\kern-1pt}}}}}1
{⁻}{{\ensuremath{^{-}}}}1
{▸}{{\ensuremath{\blacktriangleright}}}1

{∧}{{\color{symbolcolor}\ensuremath{\wedge}}}1
{∨}{{\color{symbolcolor}\ensuremath{\vee}}}1
{¬}{{\color{symbolcolor}\ensuremath{\neg}}}1
{⊢}{{\color{symbolcolor}\ensuremath{\vdash}}}1

%{⟨}{{\ensuremath{\left\langle}}}1
%{⟩}{{\ensuremath{\right\rangle}}}1
{⟨}{{\ensuremath{\langle}}}1
{⟩}{{\ensuremath{\rangle}}}1

{↦}{{\color{symbolcolor}\ensuremath{\mapsto}}}1
{→}{{\color{symbolcolor}\ensuremath{\rightarrow}}}1
{←}{{\color{symbolcolor}\ensuremath{\leftarrow}}}1
{↔}{{\color{symbolcolor}\ensuremath{\leftrightarrow}}}1
{⇒}{{\color{symbolcolor}\ensuremath{\Rightarrow}}}1
{⟹}{{\color{symbolcolor}\ensuremath{\Longrightarrow}}}1
{⇐}{{\color{symbolcolor}\ensuremath{\Leftarrow}}}1
{⟸}{{\color{symbolcolor}\ensuremath{\Longleftarrow}}}1

{∩}{{\color{symbolcolor}\ensuremath{\cap}}}1
{∪}{{\color{symbolcolor}\ensuremath{\cup}}}1
{⊓}{{\color{symbolcolor}\ensuremath{\sqcap}}}1
{⊔}{{\color{symbolcolor}\ensuremath{\sqcup}}}1
{⋃}{{\color{symbolcolor}\ensuremath{\bigcup}}}1
{⋂}{{\color{symbolcolor}\ensuremath{\bigcap}}}1
{∑}{{\color{symbolcolor}\ensuremath{\sum}}}1
{⨅}{{\color{symbolcolor}\ensuremath{\bigsqcap}}}1 % use package MnSymbol to get this symbol. The symbol also exists in stmaryrd but has an incorrect size
{⨆}{{\color{symbolcolor}\ensuremath{\bigsqcup}}}1
{⊂}{{\color{symbolcolor}\ensuremath{\subset}}}1
{⊆}{{\color{symbolcolor}\ensuremath{\subseteq}}}1
{⊄}{{\color{symbolcolor}\ensuremath{\not\subset}}}1
{⊈}{{\color{symbolcolor}\ensuremath{\nsubseteq}}}1
{⊃}{{\color{symbolcolor}\ensuremath{\supset}}}1
{⊇}{{\color{symbolcolor}\ensuremath{\supseteq}}}1
{⊅}{{\color{symbolcolor}\ensuremath{\not\supset}}}1
{⊉}{{\color{symbolcolor}\ensuremath{\nsupseteq}}}1
{∈}{{\color{symbolcolor}\ensuremath{\in}}}1
{∉}{{\color{symbolcolor}\ensuremath{\notin}}}1
{∋}{{\color{symbolcolor}\ensuremath{\ni}}}1
{∌}{{\color{symbolcolor}\ensuremath{\notni}}}1
{∅}{{\color{symbolcolor}\ensuremath{\emptyset}}}1

{∖}{{\color{symbolcolor}\ensuremath{\setminus}}}1
{†}{{\color{symbolcolor}\ensuremath{\dag}}}1

{ℕ}{{\ensuremath{\mathbb{N}}}}1
{ℤ}{{\ensuremath{\mathbb{Z}}}}1
{ℝ}{{\ensuremath{\mathbb{R}}}}1
{ℚ}{{\ensuremath{\mathbb{Q}}}}1
{ℂ}{{\ensuremath{\mathbb{C}}}}1
{𝓝}{{\ensuremath{\mathcal{N}}}}1
{𝓟}{{\ensuremath{\mathcal{P}}}}1
{𝒳}{{\ensuremath{\mathcal{X}}}}1

{⌞}{{\ensuremath{\llcorner}}}1
{⌟}{{\ensuremath{\lrcorner}}}1
{⦃}{{\ensuremath{\{\!|}}}1
{⦄}{{\ensuremath{|\!\}}}}1
{∥}{{\ensuremath{\|}}}1

{₁}{{\ensuremath{_1}}}1
{₂}{{\ensuremath{_2}}}1
{₃}{{\ensuremath{_3}}}1
{₄}{{\ensuremath{_4}}}1
{₅}{{\ensuremath{_5}}}1
{₆}{{\ensuremath{_6}}}1
{₇}{{\ensuremath{_7}}}1
{₈}{{\ensuremath{_8}}}1
{₉}{{\ensuremath{_9}}}1
{₀}{{\ensuremath{_0}}}1
{ₐ}{{\ensuremath{_a}}}1
{ᵢ}{{\ensuremath{_i}}}1
{ⱼ}{{\ensuremath{_j}}}1
{ₙ}{{\ensuremath{_n}}}1
{ₘ}{{\ensuremath{_m}}}1
{ₖ}{{\ensuremath{_k}}}1
{ₜ}{{\ensuremath{_t}}}1

{¹}{{\ensuremath{^1}}}1
{ᶜ}{{\color{symbolcolor}\ensuremath{^c}}}1
{ˢ}{{\color{symbolcolor}\ensuremath{^s}}}1
{ᵐ}{{\color{symbolcolor}\ensuremath{\textsuperscript{m}}}}1
{⁻}{{\ensuremath{^{-}}}}1
{ᵒᵖ}{{\color{symbolcolor}\textsuperscript{op}}}1
{ᶠ}{{\color{symbolcolor}\ensuremath{\textsuperscript{f}}}}1

{↑}{{\color{symbolcolor}\ensuremath{\uparrow}}}1
{↓}{{\color{symbolcolor}\ensuremath{\downarrow}}}1
{⟶}{{\color{symbolcolor}\ensuremath{\longrightarrow}}}1
{⥤}{{\color{symbolcolor}\ensuremath{\Rightarrow}}}1

{...}{{\ensuremath{\ldots}}}1
{⋯}{{\color{symbolcolor}\ensuremath{\cdots}}}1

{⌊}{{\ensuremath{\lfloor}}}1
{⌋}{{\ensuremath{\rfloor}}}1
{⌈}{{\ensuremath{\lceil}}}1
{⌉}{{\ensuremath{\rceil}}}1

{⇂}{{\ensuremath{\downharpoonright}}}1

{‹}{{\guilsinglleft}}1
{›}{{\guilsinglright}}1

{Σ}{{\color{symbolcolor}\ensuremath{\Sigma}}}1
{Π}{{\color{symbolcolor}\ensuremath{\Uppi}}}1 %\mathrm{\Uppi}
{∀}{{\color{symbolcolor}\ensuremath{\forall}}}1
{∃}{{\color{symbolcolor}\ensuremath{\exists}}}1
{λ}{{\color{symbolcolor}\ensuremath{\uplambda}}}1
{\$}{{\color{symbolcolor}\$}}1

{:}{{\color{symbolcolor}:}}1
{|}{{\color{symbolcolor}|}}1
{=}{{\color{symbolcolor}=}}1
{<}{{\color{symbolcolor}<}}1
{>}{{\color{symbolcolor}>}}1
{<|>}{{\color{symbolcolor}<|>}}1
{+}{{\color{symbolcolor}+}}1
{*}{{\color{symbolcolor}\ensuremath{{}^{*}}}}1
{\#}{{\color{keywordcolor}\#}}1, % is usually part of a keyword, like #check


% Comments
%comment=[s][\itshape \color{commentcolor}]{/-}{-/},
morecomment=[s][\color{commentcolor}]{/-}{-/},
morecomment=[l][\itshape \color{commentcolor}]{--},

% Spaces are not displayed as a special character
showstringspaces=false,

% keep spaces
keepspaces=true,

% String delimiters
morestring=[b]",
morestring=[d]’,

% Size of tabulations
tabsize=3,

% Enables ASCII chars 128 to 255
extendedchars=false,

% Case sensitivity
sensitive=true,

% Automatic breaking of long lines
breaklines=true,
breakatwhitespace=true,

% Default style fors listingsred
basicstyle=\ttfamily\small,

% Position of captions is bottom
captionpos=b,

% Full flexible columns
columns=[l]fullflexible,


% Style for (listings') identifiers
identifierstyle={\ttfamily\color{black}},
% Note : highlighting of Coq identifiers is done through a new
% delimiter definition through an lstset at the beginning of the
% document. Don't know how to do better.

% Style for declaration keywords
keywordstyle=[1]{\ttfamily\color{keywordcolor}},

% Style for sorts
keywordstyle=[2]{\ttfamily\color{sortcolor}},

% Style for tactics keywords
keywordstyle=[3]{\ttfamily\color{tacticcolor}},

% Style for attributes
keywordstyle=[4]{\ttfamily\color{attributecolor}},

% Style for strings
stringstyle=\ttfamily,

% Style for comments
commentstyle={\ttfamily\footnotesize },

}