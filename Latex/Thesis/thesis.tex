\documentclass[paper=a4, fontsize=11pt, BCOR=13mm, DIV=13, headinclude, toc=index, toc=bibliography, english, twoside]{scrreprt}
% Die verwendete Dokumentenklasse ist scrreprt. Die verwendeten Optionen sind:
%
% paper=a4              Papier ist a4.
% fontsize=11pt         Schrifgröße ist 11.
% DIV=13                Das Papier wird in d viele Spalten und d' viele Zeilen eingeteilt. Die Werte werden aus DIV berechnet.
% BCOR=1cm              Definiert den Patz, der auf der Innenseite beim Binden verloren geht.
% headinclude           sorgt dafür, dass genug Platz für die Header vorhanden ist.
% toc=index             legt im Inhaltsverzeichnis einen Eintrag für das Stichwortverzeichnis an.
% toc=bibliography      legt im Inhaltsverzeichnis einen Eintrag für das Literaturverzeichnis an.
% english               englische Worte wie "Chapter" und "References".
% twoside               Beidseitiges Dokument, wie in einem Buch.


%If you don't want this fancyheaders comment out lines 17 to 24
% Fancyheader
\usepackage{fancyhdr}                   % Wie der Name schon sagt, um fancy Header zu generieren.
\pagestyle{fancy}                       % Fancy Header sollen angezeigt werden
\renewcommand{\sectionmark}[1]{\markright{\thesection.\ #1}{}}    % Verhindert dass rightmark ausschließlich Grußbuchstaben benutzt
\fancyhead[LE,RO]{\rightmark}           % Links bei geraden und rechts bei ungeraden Seitenzahlen soll der Name der Section stehen.
\fancyhead[LO,RE]{}                     % Links bei ungeraden und rechts bei geraden Seitenzahlen soll nichts stehen.
\fancyfoot[C]{}                         % Keine mittigen Seitenzahlen
\fancyfoot[LE,RO]{\thepage}             % Seitenzahlen unten in die jeweilige äußere Ecke





\setcounter{secnumdepth}{3}     % Nummerierungstiefe (chapter, section, subsection, ...).
\setcounter{tocdepth}{3}        % Nummerierungstiefe im Inhaltsverzeichn is.

\usepackage[linesnumbered,ruled,vlined]{algorithm2e}    % Algorithmen setzen.
\usepackage{amsmath,amssymb,amsthm,amsfonts,amsbsy,latexsym}    % "Notwendige" AMS-Math Pakete.
\usepackage{array}                      % Bessere Tabellen.
\renewcommand{\arraystretch}{1.15}      % Tabellen bekommen ein wenig mehr Platz.
\usepackage{bbm}                        % Dicke 1.
\usepackage[utf8]{inputenc}             % utf8 als Eingabeformat.
\usepackage[backend=biber, style=alphabetic]{biblatex}  % Gute Erweiterung zu bibtex, Wird für Referenzen benutzt.
\bibliography{thesis}   % Die verwendeten Referenzen (.bib-Datei)
\usepackage[hypcap]{caption}            % Damit Hyperrefs bei der figure-Umgebung auf die Figure zeigt statt auf die Caption.
\usepackage[nodayofweek]{datetime}                   % Um \today einzustellen.
\newdateformat{mydate}{\THEDAY{}th \monthname{} \THEYEAR{}}
\usepackage{diagbox}                    % Diagonale in Tabellen.
\usepackage{enumitem}                   % Zum Ändern der Nummerierungsumgenung 'enumerate'
\setlist[enumerate,1]{label=(\roman*)}  % Aufzählungen sind vom Typ 'Klammer auf; kleine römische Zahl; Klammer zu'
\usepackage[T1]{fontenc}                % Bessere Schrift
\usepackage{ifthen}                     % Zum checken ob Parameter leer sind.
\usepackage{lmodern}                    % Bessere Schrift
\usepackage{listings}                   % Code Listings.
\usepackage{mathtools}                  % Subscript unter Summen behandeln. Der Befehl lautet \mathclap.
\usepackage{makeidx}                    % Stichwortverzeichnis.
\makeindex                              % Stichwortverzeichnis erstellen.
\renewcommand{\indexname}{Index}        % Name des Index definieren.
\usepackage{multirow}                   % In Tabellen mehrere Zeilen zu einer machen.
\usepackage{rotating}                   % Um Figures zu drehen.
\usepackage{scrhack}                    % Verbessert die Zusammenarbeit von KOMA mit anderen Paketen (z.B, listing).
\usepackage{stackrel}                   % Symbole übereinander stapeln.
\usepackage[dvipsnames]{xcolor}         % Gefärbter Text und so.
\usepackage{tikz}                       % Graphen und kommutative Diagramme. Muss nach xcolor eingebunden werden.
\usepackage{tikz-cd}                    % Kommutative Diagramme.
\usepackage{transparent}                % Braucht mal manchmal für inkscape bilder.
\usetikzlibrary{patterns}               % Zu malen von schraffierten Flächen.

\graphicspath{{pictures/}}              % Pfad in dem die mit Inkscape erstellen Bilder liegen (relativ zum Hauptverzeichnis).

% Workaround, damit keine unnötigen Leerzeichen entstehen.
\let\oldindexdefn\index
\renewcommand*{\index}[1]{\oldindexdefn{#1}\ignorespaces}
\let\oldlabeldefn\label
\renewcommand*{\label}[1]{\oldlabeldefn{#1}\ignorespaces}

% Workaround, Linebreak nach ldots erlaubt.
\newcommand{\origldots}{}
\let\origldots\ldots
\renewcommand{\ldots}{\allowbreak\origldots}

% to get upright greek letters
\usepackage{upgreek}

% To break urls correctly in bibliography
\setcounter{biburllcpenalty}{7000}
\setcounter{biburlucpenalty}{8000}

% to get rid of an error
\usepackage{bookmark}

%colors for lean syntax highlighting
\usepackage{color}
\definecolor{keywordcolor}{rgb}{0.7, 0.1, 0.1}   % red
\definecolor{tacticcolor}{rgb}{0.0, 0.1, 0.6}    % blue
\definecolor{commentcolor}{rgb}{0.4, 0.4, 0.4}   % grey
\definecolor{symbolcolor}{rgb}{0.0, 0.1, 0.6}    % blue
\definecolor{sortcolor}{rgb}{0.1, 0.5, 0.1}      % green
\definecolor{attributecolor}{rgb}{0.7, 0.1, 0.1} % red

% To display lean code
\def\lstlanguagefiles{lstlean.tex}
% set default language
\lstset{language=lean}


% Symbolverzeichnis
\usepackage[intoc, english]{nomencl}     % Symbolverzeichnis.
% intoc                 die Symbolliste in das Inhaltsverzeichnis aufnehmen.
% english               englische Worte wie "Seite".
\renewcommand{\nomname}{Symbol Index}   % Definiert die Überschrift des Symbolverzeichnises.
\renewcommand{\nomlabelwidth}{80pt}     % Platz der einem Symbol gegönnt wird.
\newcommand{\symbolindex}[4][]{{\nomenclature[#1]{#2}{#3\ifthenelse{\equal{#4}{}}{}{ -- #4}\nomnorefpage}}\ignorespaces}    % Verbesserte Version von "\nomenclature". Erzeugt Symbol Beschreibung - Referenz.
\renewcommand*{\nompreamble}{\markright{\nomname}}    % Workaround: Fancyhdr schreibt im Symbolverzeichnis sonst den Namen des leztztes Kapitels.
\makenomenclature                       % Symbolverzeichnis erstellen.

% Anklickbare Referenzen (letztes eingebundenes Paket)
\usepackage{hyperref}                   % Referenzen innerhalb des Dokuments anklickbar machen. Achtung: Muss das letzte Paket im Präambel sein.
\hypersetup{                            % Optionen von hyperref Einstellen.
    colorlinks=true,                    % gefärbte Links an Stelle von Boxen.
    linkcolor=black,                     % Farbe interner Links.
    citecolor=black,                     % Farbe von Referenzen.
    urlcolor=black                       % Farbe von Internetlinks.
}

\usepackage{MA_Titlepage}

%Name of the author of the thesis 
\authornew{Hannah Scholz}
%Date of birth of the Author
\geburtsdatum{\formatdate{21}{7}{2003}}
%Place of Birth
\geburtsort{Bonn, Germany}
%Date of submission of the thesis
\date{\formatdate{22}{8}{2024}}

%Name of the Advisor
% z.B.: Prof. Dr. Peter Koepke
\betreuer{Advisor: Prof. Dr. Floris van Doorn}
%name of the second advisor of the thesis
\zweitgutachter{Second Advisor: Prof. Dr. Philipp Hieronymi}

%Name of the Insitute of the advisor
%z.B.: Mathematisches Institut
%\institut{Institut XYZ}
\institut{Mathematisches Institut}
%\institut{Institut f\"ur Angewandte Mathematik}
%\institut{Institut f\"ur Numerische Simulation}
%\institut{Forschungsinstitut f\"ur Diskrete Mathematik}
%Title of the thesis 
\title{Formalisation of CW-complexes}
%Do not change!
\ausarbeitungstyp{Bachelor's Thesis  Mathematics}

%       Theoreme
\theoremstyle{definition}               % Name: dick            Text: normal.
\newtheorem{defi}{Definition}[section]  % Der Zähler ist defi = Sectionzähler.1 . Sectionzähler soll bei Benutzung von defi nicht erhöht werden.
\newtheorem*{defi*}{Definition}
\newtheorem{example}[defi]{Example}
\newtheorem{notation}[defi]{Notation}
\newtheorem{rem}[defi]{Remark}
\newtheorem{defcor}[defi]{Definition/Corollary}
\newtheorem{defprop}[defi]{Definition/Proposition}
\newtheorem{defthm}[defi]{Definition/Theorem}

\theoremstyle{plain}
\newtheorem*{conj}{Conjecture}
\newtheorem{cor}[defi]{Corollary}
\newtheorem{lem}[defi]{Lemma}
\newtheorem{prop}[defi]{Proposition}
\newtheorem*{prop*}{Proposition}
\newtheorem{thm}[defi]{Theorem}
\newtheorem*{thm*}{Theorem}


%       Makros
\newcommand{\bA}{\mathbb{A}}
\newcommand{\bB}{\mathbb{B}}
\newcommand{\bC}{\mathbb{C}}
\newcommand{\bD}{\mathbb{D}}
\newcommand{\bE}{\mathbb{E}}
\newcommand{\bF}{\mathbb{F}}
\newcommand{\bG}{\mathbb{G}}
\newcommand{\bH}{\mathbb{H}}
\newcommand{\bI}{\mathbb{I}}
\newcommand{\bJ}{\mathbb{J}}
\newcommand{\bK}{\mathbb{K}}
\newcommand{\bL}{\mathbb{L}}
\newcommand{\bM}{\mathbb{M}}
\newcommand{\bN}{\mathbb{N}}
\newcommand{\bO}{\mathbb{O}}
\newcommand{\bP}{\mathbb{P}}
\newcommand{\bQ}{\mathbb{Q}}
\newcommand{\bR}{\mathbb{R}}
\newcommand{\bS}{\mathbb{S}}
\newcommand{\bT}{\mathbb{T}}
\newcommand{\bU}{\mathbb{U}}
\newcommand{\bV}{\mathbb{V}}
\newcommand{\bW}{\mathbb{W}}
\newcommand{\bX}{\mathbb{X}}
\newcommand{\bY}{\mathbb{Y}}
\newcommand{\bZ}{\mathbb{Z}}

\newcommand{\pr}[2]{\text{pr}_{#1}(#2)} %projection
\newcommand{\compl}[1]{#1^{c}} %complement
\newcommand{\id}{\text{id}} %identity
\newcommand{\preim}[2]{#1^{-1}(#2)} %preimage
\newcommand{\interior}[1]{\text{int}(#1)} %interior
\newcommand{\closure}[1]{\overline{#1}} %closure
\newcommand{\boundary}{\partial} %boundary
\newcommand{\restrict}[2]{\ensuremath{\left.#1\right|_{#2}}} %restriction
\newcommand{\seqclosure}[1]{\text{scl}(#1)} %sequential closure
\newcommand{\oball}[2]{\text{B}_{#1}(#2)} %open ball
\newcommand{\closedCell}[2]{\closure{e}_{#2}^{#1}} %closed n-cell
\newcommand{\openCell}[2]{e_{#2}^{#1}} %open n-cell
\newcommand{\cellFrontier}[2]{\partial e_{#2}^{#1}} %edge n-cell
\newcommand{\closedCellf}[2]{\closure{f}_{#2}^{#1}} %closed n-cell
\newcommand{\openCellf}[2]{f_{#2}^{#1}} %open n-cell
\newcommand{\cellFrontierf}[2]{\partial f_{#2}^{#1}} %edge n-cell
\newcommand{\norm}[1]{\left\lVert#1\right\rVert} %norm
\newcommand{\maximum}[2]{\text{max}(#1, #2)} %maximum



\begin{document}
% Titelseite
\maketitle              % Titelseite ausgeben
\setcounter{page}{3}    % Die Titelseite und die darauffolgende leere Seite sollen gefälligst Seite 1 und 2 sein.
\tableofcontents        % Inhaltsverzeichnis ausgeben


%mathematics of CW-complexes
\cleardoublepage 
\chapter{The mathematics of CW-complexes}

\section{Definition and basic properties of a CW-complex}

The following is our definition of CW-complexes in Lean:

\begin{lstlisting}
class CWComplex.{u} {X : Type u} [TopologicalSpace X] (C : Set X) where
  cell (n : ℕ) : Type u
  map (n : ℕ) (i : cell n) : PartialEquiv (Fin n → ℝ) X
  source_eq (n : ℕ) (i : cell n) : (map n i).source = closedBall 0 1
  cont (n : ℕ) (i : cell n) : ContinuousOn (map n i) (closedBall 0 1)
  cont_symm (n : ℕ) (i : cell n) : ContinuousOn (map n i).symm (map n i).target
  pairwiseDisjoint' :
    (univ : Set (Σ n, cell n)).PairwiseDisjoint (fun ni ↦ map ni.1 ni.2 '' ball 0 1)
  mapsto (n : ℕ) (i : cell n) : ∃ I : Π m, Finset (cell m),
    MapsTo (map n i) (sphere 0 1) (⋃ (m < n) (j ∈ I m), map m j '' closedBall 0 1)
  closed' (A : Set X) (asubc : A ⊆ C) : IsClosed A ↔ ∀ n j, IsClosed (A ∩ map n j '' closedBall 0 1)
  union' : ⋃ (n : ℕ) (j : cell n), map n j '' closedBall 0 1 = C
\end{lstlisting}

The \lstinline|.{u}| is a way to fix a universe level so that our definition of a CW-complex does not depend on a number of different universe levels: The one of \lstinline{X} and the one of \lstinline{cell n} for every $n \in \bN$.
\lstinline{cell (n : ℕ)} represents the indexing set that we called $I_n$ in definition \ref{defi:CWComplex2}. \lstinline{map (n : ℕ) (i : cell n)} represent what we called $Q_i^n$ in that definition. 

\lstinline{Fin n} is the set containing $n$ natural numbers starting at 0. 
\lstinline{Fin n → ℝ} is one way to express $\bR^n$ in Lean. 
\lstinline{PartialEquiv} is a structure defined in mathlib as follows: 

\begin{lstlisting}
structure PartialEquiv (α : Type*) (β : Type*) where
  toFun : α → β
  invFun : β → α
  source : Set α
  target : Set β
  map_source' : ∀ ⦃x⦄, x ∈ source → toFun x ∈ target
  map_target' : ∀ ⦃x⦄, x ∈ target → invFun x ∈ source
  left_inv' : ∀ ⦃x⦄, x ∈ source → invFun (toFun x) = x
  right_inv' : ∀ ⦃x⦄, x ∈ target → toFun (invFun x) = x
\end{lstlisting}

It bundles two maps and two sets that get mapped to each other by the respective maps. 
Restricting the maps to these sets yields two maps that are the inverse of each other. 
We use this instead of a similar construction called \lstinline{Equiv} for bijections to avoid explicitly having to deal with restrictions. 
The brackets \lstinline{⦃⦄} are similar to the curly brackets and are used here since \lstinline{x} can be inferred from the left sides of the implications.

The property \lstinline{source_eq} specifies the source of the \lstinline{PartialEquiv}. 
\lstinline{cont} and \lstinline{cont_symm} make the bijection into a homeomorphism giving us property (i) of definition \ref{defi:CWComplex2}.
The property \lstinline{pairwiseDisjoint'} corresponds to property (ii) of definition \ref{defi:CWComplex2}. 
We are adding the prime to its name because we will later see a lemma called \lstinline{pairwiseDisjoint} that we prefer to be used. 
\lstinline{fun a : α ↦ f a} for \lstinline{f : α → β} is a way to construct a map.

\lstinline{mapsto} is the equivalent of property (iii) of the definition of a CW-complex. 
The \lstinline{Π} defines a dependent function type which we discussed in section \ref{sec:typetheory}.
\lstinline{Finset α} is the type of all finite sets in a type \lstinline{α}. 
It can be imagined as a set bundled with the information that it is finite (but note that the actual definitions of \lstinline{Finset α} and \lstinline{Set α} are quite different).
\lstinline{MapsTo} is defined as
\begin{lstlisting}
def MapsTo (f : α → β) (s : Set α) (t : Set β) : Prop := 
  ∀ ⦃x⦄, x ∈ s → f x ∈ t
\end{lstlisting}

and is relatively self-explanatory. 

\lstinline{closed'} represents property (iv) of definition \ref{defi:CWComplex2} and \lstinline{union'} represents property (v). 

\medskip

We have chosen this to be a class so that we can make use of typeclass inference which we explained in section \ref{sec:implicitandtypeclass}.

There are a few things to note about this formalisation of the definition. 
First of all it does not require $X$ to be a Hausdorff space. 
This is done so that when you define a CW-complex you can choose to first define the structure in this way and later show that it is a Haussdorff space to apply lemmas about CW-complexes of which almost all will require that $X$ is Hausdorff. 
Additionally we introduce a relative component: 
Instead of defining what it means for a space to be a CW-complex we define what it means for a subspace $C$ of $X$ to be a CW-complex in $X$.
This is useful firstly to be able to work with a nicer topology: 
If you consider $S^1$ as a CW-complex and a subspace of $\bR$ you might find it easier to work with the topology on $\bR$ instead of the subspace topology. 
Secondly constructions such as attaching cells or taking disjoint unions of CW-complexes might be easier to work with if you are already working in the same overarching type.
This approach is inspired by \cite{Gonthier2013} where the authors notice that it is helpful to consider subsets of an ambient group to avoid having to work with different group operations and similar issues. 

One question that naturally arises is whether these changes to the definition preserve the notion of a CW-complex. 
Firstly note that if we choose $X$ and $C$ to be the same we recover definition \ref{defi:CWComplex2} exactly. 
Now let us think about what happens if we choose $X$ and $C$ to be different. 
Firstly this allows us to conclude that $C$ is closed: 

\begin{lem} \label{lem:Cclosed}
  Let $X$ be a Hausdorff space and and $C$ a CW-complex in $X$ as in the formalised definition. 
  Then $C$ is closed.
\end{lem}
\begin{proof}
  Since $C \subseteq C$ it is enough to show that $C \cap \closedCell{n}{i}$ is closed for every $n \in \bN$ and $i \in I_n$. 
  But by the property \lstinline{union'} we know that $C \cap \closedCell{n}{i} = \closedCell{n}{i}$ which is closed by the same argument as in the proof of lemma \ref{lem:closedCellclosed}.
\end{proof}

This indeed excludes some CW-complexes: 

\begin{example}
  Let $I \subseteq \bR$ be an open interval. 
  Then $I$ is a CW-complex in the sense of definition \ref{defi:CWComplex2}.
\end{example}
\begin{proof}
  Since $I$ is homeomorphic to $\bR$ is it by lemma \ref{lem:homeomorph} enough to show that $\bR$ admits the structure of a CW-complex. 
  As $0$-cells we choose every $z \in \bZ \subseteq \bR$. 
  As $1$-cells we choose the intervals $(z, z + 1)$ for every $z \in \bZ \subseteq \bR$.  
  Properties (i), (ii), (iii), and (v) of definition \ref{defi:CWComplex2} are easy to verify. 
  We will therefore focus on property (iv), i.e. the weak topology. 
  The forward implication follows in the same manner as in a lot of other proofs. 
  Let us thus move on to the backwards direction. 
  Take $A \subseteq \bR$ and assume that $A \cap [z, z + 1]$ is closed for all $z \in \bZ$. 
  We now need to show that $A$ is closed.
  It is wellknown that $\bR$ is a metric space and by lemma \ref{lem:metricissequential} it is in particular a sequential space. 
  It is therefore enough to show that for every convergent sequence $(a_n)_{n \in \bN} \subseteq A$ the limit point is also in $A$. 
  Take an arbitrary convergent $(a_n)_{n \in \bN} \subseteq A$. 
  We call the limit point $a$.
  Then there exists a $z \in \bZ$ such that $a \in (z, z + 2)$. 
  Thus there is a subsequence $(a'_n)_{n \in \bN} \subseteq A \cap [z, z + 2]$ which obviously also converges to $a$. 
  But by assumption $A \cap [z, z + 2] = (A \cap [z, z + 1]) = (A \cap [z + 1, z + 2])$ is closed and therefore sequentially closed which gives us that $a \in A \cap [z, z + 2] \subseteq A$.
\end{proof}

But remember that our definition in Lean still allows us to view an open interval as a CW-complex in itself.

And every space that fulfils the formalised definition also fulfils definition \ref{defi:CWComplex2}: 

\begin{lem}
  Let $C$ be a CW-complex in a Hausdorff space $X$ as in the definition in the formalisation. 
  Then $C$ is a CW-complex as in definition \ref{defi:CWComplex2}. 
\end{lem}
\begin{proof}
  Properties (i), (ii), (iii) and (v) of definition \ref{defi:CWComplex2} are immediate. 
  Thus let us look at property (iv). 
  We assume that 
  \[A \subseteq C \text{ is closed in } X \iff \closedCell{n}{i} \cap A \text{ is closed in } X \text{ for all } n \in \bN \text{ and } i \in I_n\]
  and need to show that 
  \[A \subseteq C \text{ is closed in } C \iff \closedCell{n}{i} \cap A \text{ is closed in } C \text{ for all } n \in \bN \text{ and } i \in I_n.\]
  It is easy to see that the forward direction is true. 
  For the backwards direction take $A \subseteq C$ such that $A \cap \closedCell{n}{i}$ is closed in $C$ for all $n \in \bN$ and $i \in I_n$. 
  That means that for every $n \in \bN$ and $i \in I_n$ there is a closed set $B_i^n \subseteq X$ such that $B^n_i \cap C = A \cap \closedCell{n}{i}$. 
  But since $C$ is closed by lemma \ref{lem:Cclosed} that means that $A \cap \closedCell{n}{i}$ was already closed for every $n \in \bN$ and $i \in I_n$. 
  Thus we are done by assumption.
\end{proof}

With that we can move onto the last important difference that our new definition has. 
While \lstinline{Fin n → ℝ} is a way to represent $\bR^n$ in Lean it does not actually carry the euclidean metric but the maximum metric.
So instead of considering closed balls we are looking at cubes which does not change our definition since the two are homeomorphic. 
We could use the euclidean metric on $\bR^n$ which would be written as \lstinline{EuclideanSpace ℝ (Fin n)} but since we are mostly arguing abstractly about CW-complexes this is unnecessary and takes up more space. 

The code in the rest of the section will always have the following assumptions: 

\begin{lstlisting}
variable {X : Type*} [t : TopologicalSpace X] [T2Space X] {C : Set X} [CWComplex C]
\end{lstlisting}

where \lstinline{T2Space X} expresses that \lstinline{X} is a Hausdorff space.

We also want to define some notation in Lean. 
Just as we defined $\closedCell{n}{i}$ to represent $Q^n_i(D_i^n)$ and similar notations in definition \ref{defi:CWComplex2} we can do the same in the formalisation: 

\begin{lstlisting}
def openCell (n : ℕ) (i : cell C n) : Set X := map n i '' ball 0 1

def closedCell (n : ℕ) (i : cell C n) : Set X := map n i '' closedBall 0 1

def cellFrontier (n : ℕ) (i : cell C n) : Set X := map n i '' sphere 0 1
\end{lstlisting}

We can now state some of the properties of our definition with this new notation. 
We restate \lstinline{pairwiseDisjoint'} in two ways:

\begin{lstlisting}
lemma pairwiseDisjoint :
    (univ : Set (Σ n, cell C n)).PairwiseDisjoint (fun ni ↦ openCell ni.1 ni.2) := ⋯

lemma disjoint_openCell_of_ne {n m : ℕ} {i : cell C n} {j : cell C m}
    (ne : (⟨n, i⟩ : Σ n, cell C n) ≠ ⟨m, j⟩) : 
    openCell n i ∩ openCell m j = ∅ := ⋯
\end{lstlisting}

The second one is especially convenient to use as the hypothesis \lstinline{ne} can often be automatically verified by a tactic called \lstinline{aesop}. 
Information on \lstinline{aesop} can be found in \cite{Limperg2023}. 

The properties \lstinline{closed'} and \lstinline{union'} can be rewritten with the new notation as follows: 

\begin{lstlisting}
lemma closed (A : Set X) (asubc : A ⊆ C) :
  IsClosed A ↔ ∀ n (j : cell C n), IsClosed (A ∩ closedCell n j) := ⋯

lemma union : ⋃ (n : ℕ) (j : cell C n), closedCell n j = C := ⋯
\end{lstlisting}

As in definition \ref{defi:CWComplex2} we also want to define notation for the $n$-skeletons. 
In the first chapter we often chose to start inductions at $-1$ to make the base case trivial. 
When formalising we want to be able to use the already defined induction principles that naturally start at $0$.
For that purpose we use an auxiliary definition called \lstinline{levelaux} that is shifted by $1$ in comparison to the usual notion of the $n$-skeleton which we call \lstinline{level}: 

\begin{lstlisting}
def levelaux (C : Set X) [CWComplex C] (n : ℕ∞) : Set X :=
  ⋃ (m : ℕ) (_ : m < n) (j : cell C m), closedCell m j

def level (C : Set X) [CWComplex C] (n : ℕ∞) : Set X :=
  levelaux C (n + 1)
\end{lstlisting}

Note that we are choosing \lstinline{n} in \lstinline{ℕ∞} which is the type of natural numbers extended by infinity which can be written as \lstinline{⊤}. 
Since \lstinline{level} is defined in terms of \lstinline{levelaux} it is often trivial to derive a lemma about \lstinline{level} from the corresponding lemma about \lstinline{levelaux}. 

We can also define what it means for a CW-complex to be finite dimensional, of finite type, or finite: 

\begin{lstlisting}
class FiniteDimensional.{u} {X : Type u} [TopologicalSpace X] (C : Set X) [CWComplex C] : Prop where
  finitelevels : ∀ᶠ n in Filter.atTop, IsEmpty (cell C n)

class FiniteType.{u} {X : Type u} [TopologicalSpace X] (C : Set X) [CWComplex C] : Prop where
  finitcellFrontiers (n : ℕ) : Finite (cell C n)

class Finite.{u} {X : Type u} [TopologicalSpace X] (C : Set X) [CWComplex C] : Prop where
  finitelevels : ∀ᶠ n in Filter.atTop, IsEmpty (cell C n)
  finitecells (n : ℕ) : Finite (cell C n)
\end{lstlisting}

Property \lstinline{finitelevels} is stated in terms of a \emph{filter} which is a concept that appears frequently in mathlib.
They are often used to describe convergence in a topological way. 
As they will not be important to this thesis we will not go into details but information on filters can be found in \cite{Bourbaki1966}. 
The property \lstinline{finitelevels} is equivalent to \lstinline{∃ a, ∀ (b : ℕ), a ≤ b → IsEmpty (cell C b)}. 

To finish of this section here are the statements of some of the main results of section \ref{sec:mathsdef}.
They correspond to the results \ref{lem:skeletonunionopenCell}, \ref{lem:Frontiersubsetopen}, \ref{lem:discretelevel0}, \ref{lem:compactintersectsonlyfinite} and \ref{lem:finiteiffcompact}.

\begin{lstlisting}
lemma iUnion_openCell_eq_level (n : ℕ∞) :
  ⋃ (m : ℕ) (_ : m < n + 1) (j : cell C m), openCell m j = level C n := ⋯

lemma cellFrontier_subset_finite_openCell (n : ℕ) (i : cell C n) : 
    ∃ I : Π m, Finset (cell C m), cellFrontier n i ⊆ 
    (⋃ (m < n) (j ∈ I m), openCell m j) := ⋯

lemma isDiscrete_level_zero {A : Set X} : IsClosed (A ∩ level C 0) := ⋯

lemma compact_inter_finite (A : Set X) (compact : IsCompact A) :
  _root_.Finite 
  (Σ (m : ℕ), {j : cell C m // ¬ Disjoint A (openCell m j)}) := ⋯

lemma compact_iff_finite : IsCompact C ↔ Finite C := ⋯
\end{lstlisting}

Where \lstinline|{a : α // P a}| is the subtype of \lstinline{α} of all \lstinline{a : α} such that \lstinline{P a} for \lstinline{P : α → Prop}.
\section{Constructions}\label{sec:mathconstructions}

In this section we will discuss how to get new CW-complexes from existing ones.
We can start with some easy ones.

\subsection{Skeletons as CW-complexes}

The $n$-skeletons of a CW-complex $X$ are again CW-complexes: 

\begin{lem} \label{lem:levelcwcomplex}
    Let $-1 \le n \le \infty$. 
    Then $X_n$ is a CW-complex together with the cells $J_m \coloneq I_m$ for $m < n + 1$ and $J_m = \varnothing$ otherwise.
\end{lem}
\begin{proof}
    We need to verify the five conditions of definition \ref{defi:CWComplex2}.
    Conditions (i), (ii) and (iii) follow directly from $X$ fulfilling these conditions and condition (v) is given by the definition of the $n$-skeleton. 
    Thus we only need to worry about condition (iv), i.e. that $X_n$ has weak topology. 
    It follows easily from lemma \ref{lem:closedCellclosed} that for a set $A \subseteq X_n$ that is closed in $X_n$ the intersection $A \cap \closedCell{m}{i}$ is closed in $X_n$ for every $m \in \bN$ and $i \in J_m$. 
    We can therefore directly consider the other direction. 
    Let $A \subseteq X_n$ be a set such that for every $m \in \bN$ and $i \in J_m$ the intersection $A \cap \closedCell{m}{i}$ is closed in $X_n$. 
    We need to show that $A$ is closed in $X_n$. 
    It suffices to show that $A$ is closed in $X$. 
    By lemma \ref{lem:closediffinteropenorclosed} we need to prove that for every $m \in \bN$ and $i \in I_m$ either $A \cap \closedCell{m}{i}$ or $A \cap \openCell{m}{i}$ is closed. 
    Let us start with the case $i \in J_m$. 
    By assumption $A \cap \closedCell{m}{i}$ is closed in $X_n$. 
    The definition of the subspace topology tells us that there exists a closed set $C \subseteq X$ such that $C \cap X_n = A \cap \closedCell{m}{i}$. 
    But since $X_n$ is closed by lemma \ref{lem:levelclosed} that means that $A \cap \closedCell{m}{i}$ is also closed in $X$. 
    So we are done for this case. 
    For the case $i \notin J_m$ notice that by lemma \ref{lem:skeletonunionopenCell} we get $A \cap \openCell{m}{i} \subseteq X_n \cap \openCell{m}{i} = (\bigcup_{l < n + 1}\bigcup_{j \in I_l}\openCell{l}{j}) \cap \openCell{m}{i} = \varnothing$ since different open cells of $X$ are disjoint. 
    The empty set is obviously closed.
\end{proof}

\subsection{Disjoint union of CW-complexes}

\subsection{Image of a homeomorphism}

Homeomorphism respect the CW-complex structure: 

\begin{lem} \label{lem:homeomorph}
    Let $X$ and $Y$ be topological spaces and $f \colon X \to Y$ a homeomorphism. 
    If $X$ is a CW-complex with indexing sets $(I_n)_{n \in \bN}$ and characteristic maps $(Q_i^n)_{n \in \bN, i \in I_n}$ then $Y$ is a CW-complex with the same indexing sets and characteristic maps $(f \circ Q_i^n)_{n \in \bN, i \in I_n}$.
\end{lem}
\begin{proof}
    Properties (ii), (iii), (v) of definition \ref{defi:CWComplex2} follow easily from the fact that $f$ is a bijection. 
    Property (i) holds since we compose the characteristic maps with a homeomorphism. 
    Let us lastly look at property (iv), i.e. the weak topology. 
    We get: 
    \begin{equation*}
        \begin{split}
            A \subseteq Y \text{ is closed} &\iff \preim{f}{A} \subseteq X \text {is closed} \\
            &\iff Q_i^n(D_i^n) \cap \preim{f}{A} = \preim{f}{(f \circ Q_i^n)(D_i^n) \cap A} \subseteq X \text{ is closed} \\ 
            &\iff (f \circ Q_i^n)(D_i^n) \cap A \subseteq Y \text{ is closed}.
        \end{split}
    \end{equation*}
\end{proof}

\subsection{Subcomplexes}

One important way to get a new CW-complex from an existing one is to consider subcomplexes which we will discuss in this section. 

Let $X$ be a CW-complex. A subcomplex of $X$ is defined as follows:

\begin{defi} \label{defi:subcomplex}
    A subcomplex of $X$ is a set $E \subseteq X$ together with a set $J_n \subseteq I_n$ for every $n \in \bN$ such that:
    \begin{enumerate}
        \item $E$ is closed.
        \item $\bigcup_{n \in \bN} \bigcup_{i \in J_n} \openCell{n}{i} = E$.
    \end{enumerate}
\end{defi}

Note that here we want $E$ to be the union of the open cells instead of the union of the closed cells as in definition \ref{defi:CWComplex2}. 
But we can prove the other version easily: 

\begin{lem} \label{lem:subcomplexunionclosed}
    Let $E \subseteq X$ be a subcomplex. 
    Then $\bigcup_{n \in \bN} \bigcup_{i \in J_n} \closedCell{n}{i} = E$.
\end{lem}
\begin{proof}
    Let $n \in \bN$ and $i \in J_n$. 
    It is enough to show that $\closedCell{n}{i} \subseteq E$. 
    By lemma \ref{lem:closureopencell} $\closedCell{n}{i} = \closure{\openCell{n}{i}}$. 
    Since $E$ is closed by property (i) $\closedCell{n}{i} \subseteq E$ is equivalent to $\openCell{n}{i} \subseteq E$ which is true by property (ii).
\end{proof}

Here are some alternative ways to define subcomplexes. 
These are taken from chapter 7.4 in \cite{Jänich2001}.
The proof that these three notions are equivalent can be found in there. 
We will just show the direction that is useful to us. 

\begin{lem} \label{lem:subcomplex2}
    Let $E \subseteq X$ and $J_n \subseteq I_n$ for $n \in \bN$ be such that 
    \begin{enumerate}
        \item For every $n \in \bN$ and $i \in I_n$ we have $\closedCell{n}{i} \subseteq E$. 
        \item $\bigcup_{n \in \bN} \bigcup_{i \in J_n} \openCell{n}{i} = E$.
    \end{enumerate}
    Then $E$ is a subcomplex of $X$.
\end{lem}
\begin{proof}
    Property (ii) in definition \ref{defi:subcomplex} is clear immediately. 
    So we only need to show that $E$ is closed. 
    We apply lemma \ref{lem:closediffinteropenorclosed} which means we only need to show that for every $n \in \bN$ and $i \in I_n$ either $E \cap \closedCell{n}{i}$ or $E \cap \openCell{n}{i}$ is closed. 
    So let $n \in \bN$ and $i \in I_n$. 
    We differentiate the cases $i \in J_n$ and $i \notin J_n$.
    For the first one notice that by property (i) $E$ can be expressed as a union of closed cells: $E = \bigcup_{m \in \bN} \bigcup_{j \in J_n} \openCell{m}{j} \subseteq \bigcup_{m \in \bN} \bigcup_{j \in J_n} \closedCell{m}{j} \subseteq E$. 
    This gives us $E \cap \closedCell{n}{i} = \closedCell{n}{i}$ which is closed by lemma \ref{lem:closedCellclosed}. 
    Now for the case $i \notin J_n$ the disjointness of the open cells of $X$ gives us that $E \cap \openCell{n}{i} = (\bigcup_{m \in \bN} \bigcup_{j \in J_n} \openCell{m}{j}) \cap \openCell{n}{i} = \varnothing$ which is obviously closed. 
\end{proof}

And here is a third way to express the property of being a subcomplex:

\begin{lem}
    Let $E \subseteq X$ and $J_n \subseteq I_n$ for $n \in \bN$ be such that 
    \begin{enumerate}
        \item $E$ is a CW-complex with respect to the cells determined by $X$ and $J_n$.
        \item $\bigcup_{n \in \bN} \bigcup_{i \in J_n} \openCell{n}{i} = E$.
    \end{enumerate}
    Then $E$ is a subcomplex of $X$.
\end{lem}
\begin{proof}
    We will show that this satisfies the properties of the construction above in lemma \ref{lem:subcomplex2}.
    Property (ii) is again immediate. 
    Property (i) combined with the definition \ref{defi:CWComplex2} of a CW-complex immediately gives us property (i) of lemma \ref{lem:subcomplex2}.
\end{proof}

Now we can show that a subcomplex is indeed again a CW-complex: 

\begin{lem}
    Let $E \subseteq X$ together with $J_n \subseteq I_n$ for every $n \in \bN$ be a subcomplex of the CW-complex $X$. 
    Then $E$ is again a CW-complex with respect to the cells determined by $J_n$ and $X$.
\end{lem}
\begin{proof}
    We show this by verifying the properties in the definition \ref{defi:CWComplex2} of a CW-complex. 
    Properties (i) and (ii) are immediate and we already covered property (v) in lemma \ref{lem:subcomplexunionclosed}.

    Let us consider property (iii) i.e. closure finiteness. 
    So let $n \in \bN$ and $i \in J_n$. 
    By closure finiteness of $X$ we know that there is a finite set $E \subseteq \bigcup_{m < n} I_n$ such that $\cellFrontier{n}{i} \subseteq \bigcup_{e \in E}e$. 
    We define $E' \coloneq \{\openCell{m}{j} \in E \mid j \in J_m\}$. 
    We want to show that $\cellFrontier{n}{i} \subseteq \bigcup_{e \in E'}e$. 
    Take $x \in \cellFrontier{n}{i}$. 
    By $\cellFrontier{n}{i} \subseteq \bigcup_{e \in E}e$ there is an $\openCell{m}{j} \in E$ such that $x \in \openCell{m}{j}$. 
    It is obviously enough to show that $j \in J_m$. 
    By lemma \ref{lem:subcomplexunionclosed} we know that $x \in \cellFrontier{n}{i} \subseteq \closedCell{n}{i} \subseteq E$.
    But since $E = \bigcup_{m' \in \bN} \bigcup_{j' \in J_{m'}} \openCell{m'}{j'}$ there is $m' \in \bN$ and $j' \in J_{m'}$ such that $x \in \openCell{m'}{j'}$. 
    We know that the open cells of $X$ are disjoint which gives us $(m, j) = (m', j')$. 
    That directly implies $j \in J_m$ which we wanted to show. 

    Lastly we need to show property (iv), i.e. that $E$ has weak topology. 
    Like in a lot of our other proofs $A \subseteq E$ being closed implies that $A \cap \closedCell{n}{i}$ is closed for every $n \in \bN$ and $i \in J_n$. 
    So now take $A \subseteq E$ such that $A \cap \closedCell{n}{i}$ is closed in $E$ for every $n \in \bN$ and $i \in J_n$. 
    We need to show that $A$ is closed in $E$.
    It is enough to show that $A$ is closed in $X$. 
    We apply lemma \ref{lem:closediffinteropenorclosed} which means we only need to show that for every $n \in \bN$ and $j \in I_n$ either $A \cap \closedCell{n}{j}$ or $A \cap \openCell{n}{j}$ is closed. 
    We look at two cases. 
    Firstly consider $j \in J_n$. 
    Then $A \cap \closedCell{n}{i}$ is closed in $E$ by assumption.
    By the definition of the subspace topology this means that there exists a closed set $B \subseteq X$ such that $A \cap \closedCell{n}{i} = E \cap B$. 
    But since $E$ is closed by assumption (i) of definition \ref{defi:subcomplex} of a subcomplex that means that $A \cap \closedCell{n}{i}$ is the intersection of two closed sets in $X$ making it also closed. 
    Now let us cover the case $j \notin J_n$. 
    This gives us $A \cap \openCell{n}{j} \subseteq E \cap \openCell{n}{j} = (\bigcup_{m \in \bN} \bigcup_{i \in J_m} \openCell{m}{i}) \cap \openCell{n}{j} = \varnothing$ where the last equality holds since the open cells of $X$ are pairwise disjoint. 
    Thus $A \cap \openCell{n}{j} = \varnothing$ which is obviously closed.
\end{proof}

Now let us look at some properties of subcomplexes: 

\begin{lem} \label{lem:unionsubcomplexes}
    A union of subcomplexes $(E_i)_{i \in \iota}$ of $X$  with indexing sets $(I_{i, n})_{i \in \iota, n \in \bN}$ is again a subcomplex of $X$ with the indexing set $\bigcup_{i \in \iota} I_{i, n}$ for every $n \in \bN$.
\end{lem}
\begin{proof}
    We show that this construction satisfies the assumptions of lemma \ref{lem:subcomplex2}. 
    Property (ii) follows easily from that the fact that each of the subcomplexes $E_i$ is the union of its open cells. 
    So let us look at property (i).
    Take $n \in \bN$ and $j \in \bigcup_{i \in \iota} I_{i, n}$. 
    Then there is a $i \in \iota$ such that $j \in I_{i, n}$. 
    With lemma \ref{lem:subcomplexunionclosed} we get 
    $\closedCell{n}{j} \subseteq \bigcup_{n \in \bN}\bigcup_{j \in I_{i, n}} \closedCell{n}{j} = E_i \subseteq \bigcup_{i \in \iota} E_i$ which means we are done.
\end{proof}

\begin{rem} \label{rem:unionfinitesubcomplexes}
    We say a subcomplex is finite when it is finite as a CW-complex.
    It is easy to see that taking a finite union of finite subcomplexes of $X$ yields again a finite subcomplex of $X$.
\end{rem}

Here are two examples of finite subcomplexes that we will need:

\begin{example} \label{example:subcomplexes} ~
    \begin{enumerate}
        \item Let $i \in I_0$. Then $\closedCell{0}{i}$ is a finite subcomplex of $X$ with the indexing sets $J_0 = \{i\}$ and $J_n = \varnothing$ for $n > 0$.
        \item Let $E$ together with the indexing sets $(J_n)_{n \in \bN}$ be a finite subcomplex of $X$ and $n \in \bN$ and $i \in I_n$ such that $\cellFrontier{n}{i}$ is included in a union of cells of $E$ of dimension less than $n$. 
        Then $E \cup \openCell{n}{i}$ together with $J'_n = J_n \cup \{i\}$ and $J'_m = J'_m$ for $m \ne n$ is a finite subcomplex of $X$.
       \end{enumerate}
\end{example}

We will omit the proofs of these examples as they are quite direct to see. 

This helps us get the following lemma: 

\begin{lem} \label{lem:cellinfinitesubcomplex}
    Let $n \in \bN$ and $i \in I_n$. 
    Then there is a finite subcomplex of $X$ such that $i$ is among its cells. 
\end{lem}
\begin{proof}
    We show this by strong induction over $n$. 
    The base case $n = 0$ is directly given by the first example in \ref{example:subcomplexes}. 
    For the induction step assume that the statement is true for all $m \le n$.
    We now need to show that it then also holds for $n + 1$. 
    By closure finiteness of $X$ there is a finite set $F$ of cells of $X$ with dimension less than $n + 1$ such that $\cellFrontier{n  + 1}{i} \subseteq \bigcup_{e \in F} \closure{e}$. 
    By induction each cell $e \in F$ is part of a finite subcomplex $E_e$ of $X$. 
    By lemma \ref{lem:unionsubcomplexes} and remark \ref{rem:unionfinitesubcomplexes} $\bigcup_{e \in F}E_e$ is again a finite subcomplex of $X$. 
    The second example in \ref{example:subcomplexes} now allows us to attach the cell $\openCell{n + 1}{i}$ to this subcomplex yielding a finite subcomplex with $\openCell{n + 1}{i}$ among its cells.
\end{proof}

\begin{cor}
    Every finite set of cells of $X$ is contained in a finite subcomplex of $X$.
\end{cor}
\begin{proof}
    Let $F$ be the set of finite cells. 
    By the above lemma \ref{lem:cellinfinitesubcomplex} each cell $e \in F$ is contained in a finite subcomplex $E_e$. 
    By lemma \ref{lem:unionsubcomplexes} and remark \ref{rem:unionfinitesubcomplexes} $\bigcup_{e \in F}E_e$ is again a finite subcomplex of $X$ and we obviously have $\bigcup_{e \in F} e \subseteq \bigcup_{e \in F}E_e$.
\end{proof}

\begin{cor}
    Let $C \subseteq X$ be compact. 
    Then $C$ is contained in a finite subcomplex of $X$.
\end{cor}
\begin{proof}
    We know from lemma \ref{lem:compactintersectsonlyfinite} and property (v) in definition \ref{defi:CWComplex2} that $C$ is contained in a finite union of cells of $X$. 
    And now the above corollary tells us that these finite cells and therefore $C$ is contained in a finite subcomplex of $X$.
\end{proof}

\subsection{Product of CW-complexes}
In this subsection we will talk about the product of CW-complexes.

\input{chapter_mathematics/section_constructions/subsection_product/subsubsection_kification.tex}
\input{chapter_mathematics/section_constructions/subsection_product/subsubsection_product.tex}

%lean and mathlib
\cleardoublepage
\chapter{Lean and mathlib}

In this chapter we will discuss some general concepts about lean and its mathematical library mathlib. 

Lean itself is a programming language that is frequently used as a theorem prover. 
It was primarily developed by Leonardo de Moura who co-funded the Lean focused research organisation that has taken on the development for five years in 2023 \cite{LeanFRO2024}. 
The currently used version is called Lean 4.
More about the technical details of Lean 4 can be found in \cite{deMoura2021}.

Lean is known for its extensive mathematical library called \emph{mathlib} of which the development has been largely community driven. 
Its github repository has just over 300 different contributors and multiple new pull requests every day that get approved or rejected by the 28 maintainers. 

There have been multiple large formalizations based on mathlib. 
Here are two examples: 
In the \emph{Liquid Tensor Experiment}, a challenge given to the lean community by Peter Scholze, Johan Commelin, Adam Topaz and a number of other contributors formalised a theorem by Peter Scholze and Dustin Clausen from condensed mathematics \cite{Commelin2022}.
Floris van Doorn, Patrick Massot and Oliver Nash have formalised the existence of sphere eversions, a concept from differential topology, showing that geometric areas of mathematics can also be successfully formalised in lean \cite{vanDoorn2023}. 

There are also several large scale ongoing projects of which we present two examples again: 
Floris van Doorn is currently leading a a formalisation of a generalisation of Carleson's theorem, a theorem from fourier analysis, by Christoph Thiele and his group \cite{Becker2024}.
Additionally there is a project lead by Kevin Buzzard that aims to reduce the famous Fermat's Last Theorem to mathematical facts already known by mathematicians in the 1980s, a starting point similar to that of Andrew Wiles and Richard Taylor, who first proved this theorem in 1995 \cite{Buzzard2024}. 

\section{The type theory of Lean}
\label{sec:typetheory}

\emph{Type theory} was first proposed by \Citeauthor{Russell1908} in \citeyear{Russell1908} \cite{Russell1908} as a way to axiomatise mathematics and resolve the paradoxes (most famously Russell's paradox) that were discussed at the time. 
While type theory has lost its relevance as a foundation of mathematics to set theory it has since been studied in both mathematics and computer science. 
It was first used in formal mathematics in 1967 in the formal language \emph{AUTOMATH}. 
More about the history of type theory can be found in \cite{Kamareddine2004}. 
Discussions of type theory in mathematics and especially its connections to homotopy theory forming the new area of \emph{homotopy type theory} can be found in \cite{hottbook}.
We will now focus on the type theory as used in Lean.
A detailed account of that topic can be found in \cite{Carneiro2019}. 
The following short discussion is be based on \cite{Avigad2024}. 

Lean uses what is called a \emph{dependent type theory}.
In type theory every object has a type. 
A type can for example be the natural numbers which we write in Lean as \lstinline{Nat} or \lstinline{ℕ} or propositions which is written as \lstinline{Prop}.
To assert that \lstinline{n} is a natural number or that \lstinline{p} is a proposition we write \lstinline{n : ℕ} and \lstinline{p : Prop}. 
Proofs of a proposition \lstinline{p} also form a type written as \lstinline{Proof p}.
If you want to say that \lstinline{hp} is a proof of \lstinline{p} then you can simply write \lstinline{hp : p}.
Something to note about proofs in Lean is that contrary to other type theories the type theory of Lean has \emph{proof irrelevance} which means that two proofs of a proposition \lstinline{p} are by definition assumed to be the same.

Even types themselves are types. 
In Lean the type of natural numbers \lstinline{ℕ} has the type \lstinline{Type}. 
The type of propositions \lstinline{Prop} is also of type \lstinline{Type}.
As to not run into Russel's paradox there is a hierarchy of types. 
The type of \lstinline{Type} is \lstinline{Type 1}, the type of \lstinline{Type 1} is \lstinline{Type 2} and so on.
These are called type-universes. 
The notation \lstinline{α : Type*} is a way of stating that \lstinline{α} is a type in an arbitrary universe.

There are a few ways to construct new types from existing ones. 
Some of them are very similar to constructions on sets such as the cartesian product of types written as \lstinline{α × β} or the type of functions from \lstinline{α} to \lstinline{β} written as \lstinline{α → β} where \lstinline{α} and \lstinline{β} are types.
Since these are quite self-explanatory we will not go into more detail.
We will now mainly discuss constructions that do not fulfil this criterium. 
A first example is the sum type of two types \lstinline{α} and \lstinline{β} written as \lstinline{α ⊕ β} which is the equivalent to a disjoint union of sets. 

The next two examples explain why this type theory is a dependent type theory:
If we have a type \lstinline{α} and for every \lstinline{a : α} a type \lstinline{β a} (i.e. \lstinline{β} is a function assigning a type to every \lstinline{a : α}) then we can construct the pi type or dependent function type written as \lstinline{(a : α) → β} or \lstinline{Π (a : α), β a}. 
The dependent version of the cartesian product is called a sigma type and can be written as \lstinline{(a : α) × β a} or \lstinline{Σ a : α, β a} for \lstinline{α} and \lstinline{β} the same way as above.

\section{Implicit variables and type class inference}

A crucial factor that makes lean more comfortable to use and makes the formalisation process feel closer to doing mathematics on paper is its use of \emph{implicit variables} and \emph{typeclass inference}. 
We will explain both of these concepts in this section. 

First let us discuss implicit variables based on \cite{Avigad2024}. 
One way that we could define continuity in lean is the following: 

\begin{lstlisting}
structure Continuous' (X Y : Type*) (t : TopologicalSpace X) 
    (s : TopologicalSpace Y) (f : X → Y) : Prop where
  isOpen_preimage : ∀ s, IsOpen s → IsOpen (f ⁻¹' s)
\end{lstlisting}

But now if we are given two types \lstinline{X} and \lstinline{Y} with topologies \lstinline{s} and \lstinline{t} respectively and a map \lstinline{f : X → Y}, the statement that the map \lstinline{f} is continuous would be expressed in the following way: 

\begin{lstlisting}
    Continuous' X Y t s f
\end{lstlisting}

which is a lot longer than we would write this on paper. 

One thing that we can notice is that the types \lstinline{X} and \lstinline{Y} are contained in the definition of \lstinline{f} which means that lean should be able to find that information itself. 
To tell lean to do that you can replace the variables by underscores: 

\begin{lstlisting}
Continuous' _ _ t s f
\end{lstlisting}

Since this should always possible to be inferred we can change the definition to already specify that the two types should be clear from the context. 
We use curly brackets to do this: 

\begin{lstlisting}
structure Continuous'' {X Y : Type*} (t : TopologicalSpace X) 
    (s : TopologicalSpace Y) (f : X → Y) : Prop where
  isOpen_preimage : ∀ s, IsOpen s → IsOpen (f ⁻¹' s)
\end{lstlisting}

which enables us to write continuity like this: 

\begin{lstlisting}
    Continuous'' t s f
\end{lstlisting}

This is already a lot shorter than what we had above but there is still room for improvement as on paper you would probably just write "$f$ is continuous" since in most contexts $X$ and $Y$ will only have one specified topology each that can be inferred by the reader. 
The same thing is also true in lean and we can achieve this by typeclass inference.
Typeclasses were first invented by \Citeauthor{Wadler1989} in \cite{Wadler1989} to be used in the programming language Haskell. 
They are a way to overload operations for various different types. 
For example you might want to write code that works for all types that have a topology. 
In lean this is possible by just stating that you input type \lstinline{X} is part of the typeclass \lstinline{TopologicalSpace}. 
You can specify that something ia a typeclass with the keyword \lstinline{class}. 
The definition of the typeclass of topological spaces in mathlib looks like this

\begin{lstlisting}
class TopologicalSpace (X : Type*) where
  protected IsOpen : Set X → Prop
  protected isOpen_univ : IsOpen univ
  protected isOpen_inter : ∀ s t, IsOpen s → IsOpen t → IsOpen (s ∩ t)
  protected isOpen_sUnion : ∀ s, (∀ t ∈ s, IsOpen t) → IsOpen (⋃₀ s)
\end{lstlisting}

Let us first explain what this code means: 
The keyword \lstinline{protected} means that these properties should not be accessed directly because there are lemmas that should be used instead. 
\lstinline{Set X} is the type that consists of all sets of elements of \lstinline{X}.
Thus the line \lstinline{protected IsOpen : Set X → Prop} expresses that \lstinline{IsOpen} is a property that can be assigned to a set in \lstinline{X}.
The rest of the lines discuss the properties of a topology.
\lstinline{univ} is the set that is composed of all elements of \lstinline{X} and \lstinline{⋃₀ s} is the union over the set \lstinline{s}. 
All of these explanations are not actually relevant to typclasses, they are just for your understanding of the above code. 

Typeclasses are also expected to be inferred automatically. 
Local instances of these typeclasses can written with square brackets which tells lean to infer these automatically.

We can now look at the version of continuity that is almost identical to that of mathlib: 

\begin{lstlisting}
structure Continuous {X Y : Type*} [t : TopologicalSpace X]
    [s : TopologicalSpace Y] (f : X → Y) : Prop where
  isOpen_preimage : ∀ s, IsOpen s → IsOpen (f ⁻¹' s)
\end{lstlisting}

which enables us to write that \lstinline{f} is continuous in the context explained above as follows:

\begin{lstlisting}
Continuous f
\end{lstlisting}

When you define a class you can then define instances of that class to be inferred whenever you talk you talk about a type with that instance. 

\cleardoublepage
\appendix
\chapter*{Appendix}
\addcontentsline{toc}{chapter}{Appendix}
\renewcommand{\thesection}{\Alph{section}}
\section{a}
\section{b}

% Symbolverzeichnis
\cleardoublepage        % Auch diese sollen auf der rechten Seite beginnen
\printnomenclature      % Symbolverzeichnis ausgeben

% Stichwortverzeichnis
\cleardoublepage        % Auch diese sollen auf der rechten Seite beginnen
\printindex             % Stichwortverzeichnis ausgeben

% Referenzen
\nocite{*}              % Alle Einträge der Bib-Datei sollen in die Referenzen
\cleardoublepage        % Auch diese sollen auf der rechten Seite beginnen
\printbibliography      % Bibliographie ausgeben.

\end{document}