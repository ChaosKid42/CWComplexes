\documentclass[paper=a4, fontsize=11pt, BCOR=13mm, DIV=13, headinclude, toc=index, toc=bibliography, english, twoside]{scrreprt}
% Die verwendete Dokumentenklasse ist scrreprt. Die verwendeten Optionen sind:
%
% paper=a4              Papier ist a4.
% fontsize=11pt         Schrifgröße ist 11.
% DIV=13                Das Papier wird in d viele Spalten und d' viele Zeilen eingeteilt. Die Werte werden aus DIV berechnet.
% BCOR=1cm              Definiert den Patz, der auf der Innenseite beim Binden verloren geht.
% headinclude           sorgt dafür, dass genug Platz für die Header vorhanden ist.
% toc=index             legt im Inhaltsverzeichnis einen Eintrag für das Stichwortverzeichnis an.
% toc=bibliography      legt im Inhaltsverzeichnis einen Eintrag für das Literaturverzeichnis an.
% english               englische Worte wie "Chapter" und "References".
% twoside               Beidseitiges Dokument, wie in einem Buch.


%If you don't want this fancyheaders comment out lines 17 to 24
% Fancyheader
\usepackage{fancyhdr}                   % Wie der Name schon sagt, um fancy Header zu generieren.
\pagestyle{fancy}                       % Fancy Header sollen angezeigt werden
\renewcommand{\sectionmark}[1]{\markright{\thesection.\ #1}{}}    % Verhindert dass rightmark ausschließlich Grußbuchstaben benutzt
\fancyhead[LE,RO]{\rightmark}           % Links bei geraden und rechts bei ungeraden Seitenzahlen soll der Name der Section stehen.
\fancyhead[LO,RE]{}                     % Links bei ungeraden und rechts bei geraden Seitenzahlen soll nichts stehen.
\fancyfoot[C]{}                         % Keine mittigen Seitenzahlen
\fancyfoot[LE,RO]{\thepage}             % Seitenzahlen unten in die jeweilige äußere Ecke





\setcounter{secnumdepth}{3}     % Nummerierungstiefe (chapter, section, subsection, ...).
\setcounter{tocdepth}{3}        % Nummerierungstiefe im Inhaltsverzeichn is.

\usepackage[linesnumbered,ruled,vlined]{algorithm2e}    % Algorithmen setzen.
\usepackage{amsmath,amssymb,amsthm,amsfonts,amsbsy,latexsym}    % "Notwendige" AMS-Math Pakete.
\usepackage{array}                      % Bessere Tabellen.
\renewcommand{\arraystretch}{1.15}      % Tabellen bekommen ein wenig mehr Platz.
\usepackage{bbm}                        % Dicke 1.
\usepackage[backend=biber, style=alphabetic]{biblatex}  % Gute Erweiterung zu bibtex, Wird für Referenzen benutzt.
\bibliography{thesis}   % Die verwendeten Referenzen (.bib-Datei)
\usepackage[hypcap]{caption}            % Damit Hyperrefs bei der figure-Umgebung auf die Figure zeigt statt auf die Caption.
\usepackage[nodayofweek]{datetime}                   % Um \today einzustellen.
\newdateformat{mydate}{\THEDAY{}th \monthname{} \THEYEAR{}}
\usepackage{diagbox}                    % Diagonale in Tabellen.
\usepackage{enumitem}                   % Zum Ändern der Nummerierungsumgenung 'enumerate'
\setlist[enumerate,1]{label=(\roman*)}  % Aufzählungen sind vom Typ 'Klammer auf; kleine römische Zahl; Klammer zu'
\usepackage[T1]{fontenc}                % Bessere Schrift
\usepackage{ifthen}                     % Zum checken ob Parameter leer sind.
\usepackage[utf8]{inputenc}             % utf8 als Eingabeformat.
\usepackage{lmodern}                    % Bessere Schrift
\usepackage{listings}                   % Code Listings.
\usepackage{mathtools}                  % Subscript unter Summen behandeln. Der Befehl lautet \mathclap.
\usepackage{makeidx}                    % Stichwortverzeichnis.
\makeindex                              % Stichwortverzeichnis erstellen.
\renewcommand{\indexname}{Index}        % Name des Index definieren.
\usepackage{multirow}                   % In Tabellen mehrere Zeilen zu einer machen.
\usepackage{rotating}                   % Um Figures zu drehen.
\usepackage{scrhack}                    % Verbessert die Zusammenarbeit von KOMA mit anderen Paketen (z.B, listing).
\usepackage{stackrel}                   % Symbole übereinander stapeln.
\usepackage[dvipsnames]{xcolor}         % Gefärbter Text und so.
\usepackage{tikz}                       % Graphen und kommutative Diagramme. Muss nach xcolor eingebunden werden.
\usepackage{tikz-cd}                    % Kommutative Diagramme.
\usepackage{transparent}                % Braucht mal manchmal für inkscape bilder.
\usetikzlibrary{patterns}               % Zu malen von schraffierten Flächen.

\graphicspath{{pictures/}}              % Pfad in dem die mit Inkscape erstellen Bilder liegen (relativ zum Hauptverzeichnis).

% Workaround, damit keine unnötigen Leerzeichen entstehen.
\let\oldindexdefn\index
\renewcommand*{\index}[1]{\oldindexdefn{#1}\ignorespaces}
\let\oldlabeldefn\label
\renewcommand*{\label}[1]{\oldlabeldefn{#1}\ignorespaces}

% Workaround, Linebreak nach ldots erlaubt.
\newcommand{\origldots}{}
\let\origldots\ldots
\renewcommand{\ldots}{\allowbreak\origldots}


% Symbolverzeichnis
\usepackage[intoc, english]{nomencl}     % Symbolverzeichnis.
% intoc                 die Symbolliste in das Inhaltsverzeichnis aufnehmen.
% english               englische Worte wie "Seite".
\renewcommand{\nomname}{Symbol Index}   % Definiert die Überschrift des Symbolverzeichnises.
\renewcommand{\nomlabelwidth}{80pt}     % Platz der einem Symbol gegönnt wird.
\newcommand{\symbolindex}[4][]{{\nomenclature[#1]{#2}{#3\ifthenelse{\equal{#4}{}}{}{ -- #4}\nomnorefpage}}\ignorespaces}    % Verbesserte Version von "\nomenclature". Erzeugt Symbol Beschreibung - Referenz.
\renewcommand*{\nompreamble}{\markright{\nomname}}    % Workaround: Fancyhdr schreibt im Symbolverzeichnis sonst den Namen des leztztes Kapitels.
\makenomenclature                       % Symbolverzeichnis erstellen.

% Anklickbare Referenzen (letztes eingebundenes Paket)
\usepackage{hyperref}                   % Referenzen innerhalb des Dokuments anklickbar machen. Achtung: Muss das letzte Paket im Präambel sein.
\hypersetup{                            % Optionen von hyperref Einstellen.
    colorlinks=true,                    % gefärbte Links an Stelle von Boxen.
    linkcolor=black,                     % Farbe interner Links.
    citecolor=black,                     % Farbe von Referenzen.
    urlcolor=black                       % Farbe von Internetlinks.
}

\usepackage{MA_Titlepage}

%Name of the author of the thesis 
\authornew{Hannah Scholz}
%Date of birth of the Author
\geburtsdatum{\formatdate{21}{7}{2003}}
%Place of Birth
\geburtsort{Bonn, Germany}
%Date of submission of the thesis
\date{\formatdate{22}{8}{2024}}

%Name of the Advisor
% z.B.: Prof. Dr. Peter Koepke
\betreuer{Advisor: Prof. Dr. Floris van Doorn}
%name of the second advisor of the thesis
\zweitgutachter{Second Advisor: Prof. Dr. X Y}

%Name of the Insitute of the advisor
%z.B.: Mathematisches Institut
%\institut{Institut XYZ}
\institut{Mathematisches Institut}
%\institut{Institut f\"ur Angewandte Mathematik}
%\institut{Institut f\"ur Numerische Simulation}
%\institut{Forschungsinstitut f\"ur Diskrete Mathematik}
%Title of the thesis 
\title{Formalisation of CW-complexes}
%Do not change!
\ausarbeitungstyp{Bachelor's Thesis  Mathematics}

%       Theoreme
\theoremstyle{definition}               % Name: dick            Text: normal.
\newtheorem{defi}{Definition}[section]  % Der Zähler ist defi = Sectionzähler.1 . Sectionzähler soll bei Benutzung von defi nicht erhöht werden.
\newtheorem*{defi*}{Definition}
\newtheorem{example}[defi]{Example}
\newtheorem{notation}[defi]{Notation}
\newtheorem{rem}[defi]{Remark}
\newtheorem{defcor}[defi]{Definition/Corollary}
\newtheorem{defprop}[defi]{Definition/Proposition}
\newtheorem{defthm}[defi]{Definition/Theorem}

\theoremstyle{plain}
\newtheorem*{conj}{Conjecture}
\newtheorem{cor}[defi]{Corollary}
\newtheorem{lem}[defi]{Lemma}
\newtheorem{prop}[defi]{Proposition}
\newtheorem*{prop*}{Proposition}
\newtheorem{thm}[defi]{Theorem}
\newtheorem*{thm*}{Theorem}


%       Makros
\newcommand{\bA}{\mathbb{A}}
\newcommand{\bB}{\mathbb{B}}
\newcommand{\bC}{\mathbb{C}}
\newcommand{\bD}{\mathbb{D}}
\newcommand{\bE}{\mathbb{E}}
\newcommand{\bF}{\mathbb{F}}
\newcommand{\bG}{\mathbb{G}}
\newcommand{\bH}{\mathbb{H}}
\newcommand{\bI}{\mathbb{I}}
\newcommand{\bJ}{\mathbb{J}}
\newcommand{\bK}{\mathbb{K}}
\newcommand{\bL}{\mathbb{L}}
\newcommand{\bM}{\mathbb{M}}
\newcommand{\bN}{\mathbb{N}}
\newcommand{\bO}{\mathbb{O}}
\newcommand{\bP}{\mathbb{P}}
\newcommand{\bQ}{\mathbb{Q}}
\newcommand{\bR}{\mathbb{R}}
\newcommand{\bS}{\mathbb{S}}
\newcommand{\bT}{\mathbb{T}}
\newcommand{\bU}{\mathbb{U}}
\newcommand{\bV}{\mathbb{V}}
\newcommand{\bW}{\mathbb{W}}
\newcommand{\bX}{\mathbb{X}}
\newcommand{\bY}{\mathbb{Y}}
\newcommand{\bZ}{\mathbb{Z}}

\newcommand{\pr}[2]{\text{pr}_{#1}(#2)} %projection
\newcommand{\compl}[1]{#1^{c}} %complement
\newcommand{\id}{\text{id}} %identity
\newcommand{\preim}[2]{#1^{-1}(#2)} %preimage
\newcommand{\interior}[1]{\text{int}(#1)} %interior
\newcommand{\closure}[1]{\overline{#1}} %closure
\newcommand{\boundary}{\partial} %boundary
\newcommand{\restrict}[2]{\ensuremath{\left.#1\right|_{#2}}} %restriction
\newcommand{\seqclosure}[1]{\text{scl}(#1)} %sequential closure
\newcommand{\oball}[2]{\text{B}_{#1}(#2)} %open ball
\newcommand{\closedCell}[2]{\closure{e}_{#2}^{#1}} %closed n-cell
\newcommand{\openCell}[2]{e_{#2}^{#1}} %open n-cell
\newcommand{\cellFrontier}[2]{\partial e_{#2}^{#1}} %edge n-cell
\newcommand{\closedCellf}[2]{\closure{f}_{#2}^{#1}} %closed n-cell
\newcommand{\openCellf}[2]{f_{#2}^{#1}} %open n-cell
\newcommand{\cellFrontierf}[2]{\partial f_{#2}^{#1}} %edge n-cell
\newcommand{\norm}[1]{\left\lVert#1\right\rVert} %norm
\newcommand{\maximum}[2]{\text{max}(#1, #2)} %maximum



\begin{document}
% Titelseite
\maketitle              % Titelseite ausgeben
\setcounter{page}{3}    % Die Titelseite und die darauffolgende leere Seite sollen gefälligst Seite 1 und 2 sein.
\tableofcontents        % Inhaltsverzeichnis ausgeben

% Einleitung
\cleardoublepage        % Kapitel immer rechts beginnen
\chapter{Introduction}
Here you have your introduction. This template is mainly based on Felix Boes Master thesis you can find it here https://github.com/felixboes/masters\_thesis/tree/master
\section{First section}
\label{introduction:first_section}
Congratulations you have created your first section
\subsection{Subsections}
If you need to divide your thesis  use subsections.\\
You can iterate this with subsub-sections if you want. If you want to do this you have to change the depth of the subsub-sections in the main.\\
If you want to create a subsection, which does not appear in the contents table use \text{"subsection*{}"}
 % for each section you can do a new document for a better organization
\section{How to cite}
\label{introduction:how_to_cite}
In this section we will discuss how to cite.\\
Just add your reference in masterthesis\_your\_name\_bibliography.bib\\
Then use this line \cite{Abhau200501} or \cite{Disertori:2006}

% Modelle für den Modulraum
\cleardoublepage        % Kapitel immer rechts beginnen
\chapter{One}
In this chapter we will learn some useful tools.\\
\section{How to cross reference}
In this section we will learn to cross reference.\\
For this just use command \ref{introduction:first_section}.  You previously need to create a label.\\
You can also reference equations in this way
\begin{equation}
\label{eq:1}
\langle v,\text{Re}A v\rangle =\langle v,U^*DUv\rangle=\langle Uv,DUv\rangle\geq \lambda_1\|Uv\|^2=\lambda_1\|v\|^2
\end{equation}

\begin{lem}
\label{lem:1}
Let $A\in \mathbb{C}^{M\times M}$ be diagonally dominant then $A$ is invertible.
\end{lem}

This equations can be accsessed by \ref{eq:1} and \ref{lem:1}
\section{How to create Index}

In this section we will learn to add elements to the  index.\\
Just use the command as in the example.
\begin{defi}
\index{vectorspace}
    A vectorspace is...
\end{defi}

\section{How to create symbol index}
In this section we will learn to add elements to the symbol index.\\
Just use the command as in the example.

\nomenclature{$\bR$}{The real numbers}


% product of CW-complexes
\cleardoublepage        % Kapitel immer rechts beginnen
\chapter{Product}
In this chapter we will talk about the product.

\section{K-spaces and the k-ification}

\begin{lem}
    Let $X$ be a k-space.
    Then the topologies of $X$ and $X_c$ coincide.
\end{lem}

\begin{lem}
    Let $X$ be an anti-compact T$_1$ space.
    Then $X_c$ has discrete topology.
\end{lem}
\begin{proof}
    Let $A \subseteq X_c$ be any set. We need to show that it is open. 
    By the definition of the k-ification it is enough to show that $A \cap C$
    is open in $C$ for every compact set $C \subseteq X$. 
    Since $X$ is anti-compact $C$ is finite.
    And by T$_1$ every finite set has discrete topology. 
    Thus $A \cap C$ is open in $C$ and $X_c$ has discrete topology.
\end{proof}

\begin{cor}
    Let $X$ be a non-discrete anti-compact T$_1$ space. 
    Then $X$ is not a k-space.
\end{cor}
\section{The product of CW-complexes}

\begin{lem}
    $(X \times Y)_c$ has weak topology,
    i.e. $A \subseteq (X \times Y)_c$ is closed iff $(Q_i^n \times P_j^m)(D^{n + m}) \cap A$ is closed for all $n, m \in \bN$, $i \in I_n$ and $j \in J_m$.
\end{lem}
\begin{proof}~
    \begin{enumerate}
        \item["$\Rightarrow$"] Since $D^{n + m}$ is compact, its image is compact and therefore closed. As the intersection of two closed sets $(Q_i^n \times P_j^m)(D^{n + m}) \cap A$ is closed as well.
        \item["$\Leftarrow$"] We know by definition of the k-ification that $A$ is closed if for every compact set $C \subseteq X \times Y$ $A \cap C$ is closed in $C$.
        Take such a compact set $C$.
        The projections $\pr{1}{C}$ and $\pr{2}{C}$ are compact as images of a compact set. 
        By ? there are finite sets $E \subseteq \{e_i^n \mid n \in \bN, i \in I_n \}$ and $F \subseteq \{f_j^m \mid m \in \bN, j \in J_m \}$ s.t $\pr{1}{C} \subseteq \bigcup_{e \in E} e$ and $\pr{2}{C} \subseteq \bigcup_{f \in F} f$.
        Thus 
        \[C \subseteq \pr{1}{C} \times \pr{2}{C} \subseteq \bigcup_{e \in E} e \times \bigcup_{f \in F} f = \bigcup_{e \in E} \bigcup_{f \in F} e \times f.\] 
        So $C$ is included in a finite union of cells of $(X \times Y)_c$. 
        Therefore 
        \[A \cap C = A \cap \left (\bigcup_{e \in E} \bigcup_{f \in F} e \times f \right )\cap C = \left (\bigcup_{e \in E} \bigcup_{f \in F} A \cap (e \times f)\right ) \cap C\] 
        is closed since by assumption $A \cap (e \times f)$ is closed for every $e$ and $f$ and the intersection is finite. Thus $A \cap C$ is in particular closed in $C$.
    \qedhere
    \end{enumerate}
\end{proof}


\cleardoublepage
\appendix
\chapter*{Appendix}
\addcontentsline{toc}{chapter}{Appendix}
\renewcommand{\thesection}{\Alph{section}}
\section{a}
\section{b}

% Weitere Symbole für das Symbolverzeichnis
\symbolindex[f]{$F$}{A topological or Riemann surface.}{}

% Symbolverzeichnis
\cleardoublepage        % Auch diese sollen auf der rechten Seite beginnen
\printnomenclature      % Symbolverzeichnis ausgeben

% Stichwortverzeichnis
\cleardoublepage        % Auch diese sollen auf der rechten Seite beginnen
\printindex             % Stichwortverzeichnis ausgeben

% Referenzen
\nocite{*}              % Alle Einträge der Bib-Datei sollen in die Referenzen
\cleardoublepage        % Auch diese sollen auf der rechten Seite beginnen
\printbibliography      % Bibliographie ausgeben.

\end{document}