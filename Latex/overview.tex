\documentclass[colorinlistoftodos]{article}
\usepackage{fontspec}
% switch to a monospace font supporting more Unicode characters
\setmonofont{FreeMono}
\usepackage{hyperref}
\usepackage{minted}
\usepackage{todonotes}
\reversemarginpar


\setminted[python]{
    linenos=true,
    breaklines=true,
    encoding=utf8,
    fontsize=\footnotesize,
    frame=lines
}
\usepackage{etoolbox}

\makeatletter
\AtBeginEnvironment{minted}{\dontdofcolorbox}
\def\dontdofcolorbox{\renewcommand\fcolorbox[4][]{##4}}
\makeatother


\makeatletter
\ifwindows
  \renewcommand{\minted@optlistcl@quote}[2]{%
    \ifstrempty{#2}{\detokenize{#1}}{\detokenize{#1="#2"}}}
\else
  \renewcommand{\minted@optlistcl@quote}[2]{%
    \ifstrempty{#2}{\detokenize{#1}}{\detokenize{#1='#2'}}}
\fi

% similar to \minted@def@optcl@switch
\newcommand{\minted@def@optcl@novalue}[2]{%
  \define@booleankey{minted@opt@g}{#1}%
    {\minted@addto@optlistcl{\minted@optlistcl@g}{#2}{}%
     \@namedef{minted@opt@g:#1}{true}}
    {\@namedef{minted@opt@g:#1}{false}}
  \define@booleankey{minted@opt@g@i}{#1}%
    {\minted@addto@optlistcl{\minted@optlistcl@g@i}{#2}{}%
     \@namedef{minted@opt@g@i:#1}{true}}
    {\@namedef{minted@opt@g@i:#1}{false}}
  \define@booleankey{minted@opt@lang}{#1}%
    {\minted@addto@optlistcl@lang{minted@optlistcl@lang\minted@lang}{#2}{}%
     \@namedef{minted@opt@lang\minted@lang:#1}{true}}
    {\@namedef{minted@opt@lang\minted@lang:#1}{false}}
  \define@booleankey{minted@opt@lang@i}{#1}%
    {\minted@addto@optlistcl@lang{minted@optlistcl@lang\minted@lang @i}{#2}{}%
     \@namedef{minted@opt@lang\minted@lang @i:#1}{true}}
    {\@namedef{minted@opt@lang\minted@lang @i:#1}{false}}
  \define@booleankey{minted@opt@cmd}{#1}%
      {\minted@addto@optlistcl{\minted@optlistcl@cmd}{#2}{}%
        \@namedef{minted@opt@cmd:#1}{true}}
      {\@namedef{minted@opt@cmd:#1}{false}}
}
\minted@def@optcl@novalue{custom lexer}{-x}
\makeatother

\newmintinline[lean]{lean4.py:Lean4Lexer}{custom lexer,bgcolor=white, fontsize=\footnotesize}
\newminted[leancode]{lean4.py:Lean4Lexer}{custom lexer,fontsize=\footnotesize}

\newcommand{\question}[1]{\todo[color=green!40]{#1}}
\newcommand{\questionsmall}[1]{\todo[color=green!40, size=\footnotesize]{#1}}
\newcommand{\questionsinline}[1]{\todo[color=green!40, inline]{#1}}
\newcommand{\comment}[1]{\todo[color=lightgray, nolist]{#1}}
\newcommand{\commentinline}[1]{\todo[inline, color=lightgray, nolist]{#1}}
\newcommand{\todonoturgent}[1]{\todo[color=yellow]{#1}}
\newcommand{\todonoturgentinline}[1]{\todo[inline, color=yellow]{#1}}

\title{Overview Code CW-Complexes}
\author{Hannah Scholz}
\date{April 2024}

\begin{document}

\maketitle

\section{Introduction}

This is a place to keep track of all the statements I already have.
I will document what still needs to be done, what I am stuck on and why and what I have questions about.

\tableofcontents

\newpage

\todototoc
\listoftodos

\newpage

\section{General}

\commentinline{Sources for types: Theorem proving in lean, Reference manual, \href{https://raw.githubusercontent.com/blanchette/logical_verification_2023/main/hitchhikers_guide.pdf}{Hitchhiker's guide to proof verification} (sections 4.6, 13.3, 13.4, 13.5)}
\questionsinline{Should there be instances for CW-Complexes? So that for example with a subcomplex you automatically get the CW-Complex.}

\section{File: auxiliary}

\begin{itemize}
  \item Lemma \emph{aux1}
\begin{leancode}
lemma aux1 (l : ℕ) {X : Type*} {s : ℕ →  Type*} (Y : (m : ℕ) → s m → Set X):
    ⋃ m, ⋃ (_ : m < Nat.succ l), ⋃ j, Y m j = 
    (⋃ (j : s l), Y l j) ∪ ⋃ m, ⋃ (_ : m < l), ⋃ j, Y m j
\end{leancode}
  \item Lemma \emph{ENat.coe\_lt\_top}
\begin{leancode}
lemma ENat.coe_lt_top {n : ℕ} : ↑n < (⊤ : ℕ∞)
\end{leancode}
  \item Lemma \emph{sphere\_zero\_dim\_empty}
\begin{leancode}
lemma sphere_zero_dim_empty {X : Type*} {h : PseudoMetricSpace (Fin 0 → X)}:
  (Metric.sphere ![] 1 : Set (Fin 0 → X)) = ∅
\end{leancode}
  \item Definition \emph{kification}
\begin{leancode}
def kification (X : Type*) := X
\end{leancode}
  \item Instance \emph{instkification}
\begin{leancode}
instance instkification {X : Type*} [t : TopologicalSpace X] : 
TopologicalSpace (kification X) where
  IsOpen A := IsOpen A := ∀ (B : t.Compacts), ∃ (C: t.Opens), A ∩ B.1 = C.1 ∩ B.1
  isOpen_univ := sorry
  isOpen_inter := sorry
  isOpen_sUnion := sorry
\end{leancode}
\end{itemize}

\section{File: Definition}

\begin{itemize}
  \item Assumption \lean /variable {X : Type*} [t : TopologicalSpace X]/
  \item Structure \emph{CWCcomplex}
\begin{leancode}
structure CWComplex.{u} {X : Type u} [TopologicalSpace X] (C : Set X) where
  cell (n : ℕ) : Type u
  map (n : ℕ) (i : cell n) : PartialEquiv (Fin n → ℝ) X
  source_eq (n : ℕ) (i : cell n) : (map n i).source = closedBall 0 1
  cont (n : ℕ) (i : cell n) : ContinuousOn (map n i) (closedBall 0 1)
  cont_symm (n : ℕ) (i : cell n) : ContinuousOn (map n i).symm (map n i).target
  pairwiseDisjoint :
    (univ : Set (Σ n, cell n)).PairwiseDisjoint (fun ni ↦ map ni.1 ni.2 '' ball 0 1)
  mapsto (n : ℕ) (i : cell n) : ∃ I : Π m, Finset (cell m),
    MapsTo (map n i) (sphere 0 1) (⋃ (m < n) (j ∈ I m), map m j '' closedBall 0 1)
  closed (A : Set X) (asubc : A ⊆ ↑C) : 
    IsClosed A ↔ ∀ n j, IsClosed (A ∩ map n j '' closedBall 0 1)
  union : ⋃ (n : ℕ) (j : cell n), map n j '' closedBall 0 1 = C
\end{leancode}
  \item Assumption \lean /variable [T2Space X] {C : Set X} (hC : CWComplex C)/
  \item Definition \emph{levelaux}
\begin{leancode}
def levelaux (n : ℕ∞) : Set X :=
  ⋃ (m : ℕ) (hm : m < n) (j : hC.cell m), hC.map m j '' closedBall 0 1
\end{leancode}
  \item Definition \emph{level}
\begin{leancode}
def level (n : ℕ∞) : Set X :=
  hC.levelaux (n + 1)
\end{leancode}
\todonoturgentinline{Add CW-Complex of finite type}
  \item Class \emph{Finite} \question{Should this be a class?}
\begin{leancode}
class Finite.{u} {X : Type u} [TopologicalSpace X] (C : Set X) (cwcomplex : CWComplex C) :
Prop where
  finitelevels : ∀ᶠ n in Filter.atTop, IsEmpty (cwcomplex.cell n)
  finitecells (n : ℕ) : Finite (cwcomplex.cell n)
\end{leancode}
  \item Structure \emph{Subcomplex}
\begin{leancode}
structure Subcomplex (E : Set X) where
  I : Π n, Set (hC.cell n)
  closed : IsClosed E
  union : E = ⋃ (n : ℕ) (j : I n), hC.map n j '' ball 0 1
\end{leancode}
  \item Class 
\begin{leancode}
class Subcomplex.Finite (E : Set X) (subcomplex: hC.Subcomplex E) : Prop where
  finitelevels : ∀ᶠ n in Filter.atTop, IsEmpty (hC.cell n)
  finitecells (n : ℕ) : _root_.Finite (hC.cell n)
\end{leancode}
  \item Lemma \emph{levelaux\_top}
\begin{leancode}
@[simp] lemma levelaux_top : hC.levelaux ⊤ = C
\end{leancode}
  \item Lemma \emph{level\_top}
\begin{leancode}
@[simp] lemma level_top : hC.level ⊤ = C
\end{leancode}
  \item Lemma \emph{iUnion\_map\_sphere\_subset\_levelaux}
\begin{leancode}
lemma iUnion_map_sphere_subset_levelaux (l : ℕ) : 
  ⋃ (j : hC.cell l), ↑(hC.map l j) '' sphere 0 1 ⊆ hC.levelaux l
\end{leancode}
  \item Lemma \emph{iUnion\_map\_sphere\_subset\_level}
\begin{leancode}
lemma iUnion_map_sphere_subset_level (l : ℕ) :
  ⋃ (j : hC.cell l), ↑(hC.map l j) '' sphere 0 1 ⊆ hC.levelaux l
\end{leancode}
  \item Lemma \emph{levelaux\_subset\_levelaux\_of\_le}
\begin{leancode}
lemma levelaux_subset_levelaux_of_le {n m : ℕ∞} (h : m ≤ n) :
  hC.levelaux m ⊆ hC.levelaux n
\end{leancode}
  \item Lemma \emph{level\_subset\_level\_of\_le}
\begin{leancode}
lemma level_subset_level_of_le {n m : ℕ∞} (h : m ≤ n) : hC.level m ⊆ hC.level n
\end{leancode}
  \item Lemma \emph{iUnion\_levelaux\_eq\_levelaux}
\begin{leancode}
lemma iUnion_levelaux_eq_levelaux (n : ℕ∞) : 
  ⋃ (m : ℕ) (hm : m < n + 1), hC.levelaux m = hC.levelaux n
\end{leancode}
  \item Lemma \emph{iUnion\_ball\_eq\_levelaux} \todonoturgent{Extraxt lemma (marked in code)}
\begin{leancode}
lemma iUnion_ball_eq_levelaux (n : ℕ∞) :
  ⋃ (m : ℕ) (hm : m < n) (j : hC.cell m), hC.map m j '' ball 0 1 = hC.levelaux n
\end{leancode}
  \item Lemma \emph{iUnion\_ball\_eq\_level}
\begin{leancode}
lemma iUnion_ball_eq_level (n : ℕ∞) :
  ⋃ (m : ℕ) (hm : m < n + 1) (j : hC.cell m), hC.map m j '' ball 0 1 = hC.level n
\end{leancode}
  \item Lemma \emph{mapsto\_sphere\_levelaux}
\begin{leancode}
lemma mapsto_sphere_levelaux (n : ℕ) (j : hC.cell n) (nnezero : n ≠ 0) : 
  MapsTo (hC.map n j) (sphere 0 1) (hC.levelaux  n)
\end{leancode}
  \item Lemma \emph{mapsto\_sphere\_level}
\begin{leancode}
lemma mapsto_sphere_level (n : ℕ) (j : hC.cell n) (nnezero : n ≠ 0) : 
  MapsTo (hC.map n j) (sphere 0 1) (hC.level (Nat.pred n))
\end{leancode}
  \item Lemma \emph{exists\_mem\_ball\_of\_mem\_levelaux}
\begin{leancode}
lemma exists_mem_ball_of_mem_levelaux {n : ℕ∞} {x : X} (xmemlvl : x ∈ hC.levelaux n) : 
  ∃ (m : ℕ) (_ : m < n) (j : hC.cell m), x ∈ ↑(hC.map m j) '' ball 0 1
\end{leancode}
  \item Lemma \emph{exists\_mem\_ball\_of\_mem\_level}
\begin{leancode}
lemma exists_mem_ball_of_mem_level {n : ℕ∞} {x : X} (xmemlvl : x ∈ hC.level n) : 
  ∃ (m : ℕ) (_ : m ≤ n) (j : hC.cell m), x ∈ ↑(hC.map m j) '' ball 0 1
\end{leancode}
  \item Lemma \emph{levelaux\_inter\_image\_closedBall\_eq\_levelaux\_inter\_image\_sphere}
\begin{leancode}
lemma levelaux_inter_image_closedBall_eq_levelaux_inter_image_sphere 
{n : ℕ∞} {m : ℕ}{j : hC.cell m} (nlem : n ≤ m) :
  hC.levelaux n ∩ ↑(hC.map m j) '' closedBall 0 1 = 
  hC.levelaux n ∩ ↑(hC.map m j) '' sphere 0 1
\end{leancode}
  \item Lemma \emph{level\_inter\_image\_closedBall\_eq\_level\_inter\_image\_sphere}
\begin{leancode}
lemma level_inter_image_closedBall_eq_level_inter_image_sphere 
{n : ℕ∞} {m : ℕ}{j : hC.cell m} (nltm : n < m) : 
  hC.level n ∩ ↑(hC.map m j) '' closedBall 0 1 = 
  hC.level n ∩ ↑(hC.map m j) '' sphere 0 1
\end{leancode}
  \item Lemma \emph{isClosed\_map\_sphere}
\begin{leancode}
lemma isClosed_map_sphere {n : ℕ} {i : hC.cell n} : IsClosed (hC.map n i '' sphere 0 1)
\end{leancode}
  \item Lemma \emph{isClosed\_inter\_sphere\_succ\_of\_le\_isClosed\_inter\_closedBall\_of\_mapsto}
  \todo[caption={Unify this lemma and the next.}, size=\footnotesize, color=yellow]
  {\textbf{Unify this lemma and the next.}
  The problem is that in one I have the type (cell n) and in the other (Set (cell n)).
  I think this will be solved as well if I can solve the subtype of a type problem in CWComplex\_ subcomplex}
\begin{leancode}
lemma isClosed_inter_sphere_succ_of_le_isClosed_inter_closedBall_of_mapsto
{A : Set X} {n : ℕ} (I : (m : ℕ) → Set (hC.cell m))
(hn : ∀ m ≤ n, ∀ (j : hC.cell m), IsClosed (A ∩ ↑(hC.map m j) '' closedBall 0 1))
(mapsto : ∀ (n : ℕ) i, ∃ I : Π m, Finset (I m),
MapsTo (hC.map n i) (sphere 0 1 : Set (Fin n → ℝ))
(⋃ (m < n) (j ∈ I m), hC.map m j '' closedBall 0 1)) :
  ∀ (j : hC.cell (n + 1)), IsClosed (A ∩ ↑(hC.map (n + 1) j) '' sphere 0 1)
\end{leancode}
  \item Lemma \emph{isClosed\_inter\_sphere\_succ\_of\_le\_isClosed\_inter\_closedBall}
\begin{leancode}
  lemma isClosed_inter_sphere_succ_of_le_isClosed_inter_closedBall
  {A : Set X} {n : ℕ}
  (hn : ∀ m ≤ n, ∀ (j : hC.cell m), IsClosed (A ∩ ↑(hC.map m j) '' closedBall 0 1)) : 
  ∀ (j : hC.cell (n + 1)), IsClosed (A ∩ ↑(hC.map (n + 1) j) '' sphere 0 1)
\end{leancode}
  \item Lemma \emph{isClosed\_map\_closedBall}
\begin{leancode}
lemma isClosed_map_closedBall (n : ℕ) (i : hC.cell n) : 
  IsClosed (hC.map n i '' closedBall 0 1)
\end{leancode}
  \item Lemma \emph{isClosed}
\begin{leancode}
lemma isClosed : IsClosed C
\end{leancode}
  \item Lemma \emph{levelaux\_succ\_eq\_levelaux\_union\_iUnion}
\begin{leancode}
lemma levelaux_succ_eq_levelaux_union_iUnion (n : ℕ) : 
  hC.levelaux (↑n + 1) = hC.levelaux ↑n ∪ ⋃ (j : hC.cell ↑n), hC.map ↑n j '' closedBall 0 1
\end{leancode}
  \item Lemma \emph{level\_succ\_eq\_level\_union\_iUnion}
\begin{leancode}
lemma level_succ_eq_level_union_iUnion (n : ℕ) : 
  hC.level (↑n + 1) = 
  hC.level ↑n ∪ ⋃ (j : hC.cell (↑n + 1)), hC.map (↑n + 1) j '' closedBall 0 1
\end{leancode}
  \item Lemma \emph{map\_closedBall\_subset\_levelaux}
\begin{leancode}
lemma map_closedBall_subset_levelaux (n : ℕ) (j : hC.cell n) : 
  (hC.map n j) '' closedBall 0 1 ⊆ hC.levelaux (n + 1)
\end{leancode}
  \item Lemma \emph{map\_closedBall\_subset\_level}
\begin{leancode}
lemma map_closedBall_subset_level (n : ℕ) (j : hC.cell n) : 
  (hC.map n j) '' closedBall 0 1 ⊆ hC.level n
\end{leancode}
  \item Lemma \emph{map\_ball\_subset\_levelaux}
\begin{leancode}
lemma map_ball_subset_levelaux (n : ℕ) (j : hC.cell n) : 
  (hC.map n j) '' ball 0 1 ⊆ hC.levelaux (n + 1)
\end{leancode}
  \item Lemma \emph{map\_ball\_subset\_level}
\begin{leancode}
lemma map_ball_subset_level (n : ℕ) (j : hC.cell n) : 
  (hC.map n j) '' ball 0 1 ⊆ hC.level n
\end{leancode}
  \item Lemma \emph{map\_ball\_subset\_complex}
\begin{leancode}
lemma map_ball_subset_complex (n : ℕ) (j : hC.cell n) : 
  (hC.map n j) '' ball 0 1 ⊆ C
\end{leancode}
  \item Lemma \emph{map\_ball\_subset\_map\_closedball}
\begin{leancode}
lemma map_ball_subset_map_closedball {n : ℕ} {j : hC.cell n} :
  hC.map n j '' ball 0 1 ⊆ hC.map n j '' closedBall 0 1
\end{leancode}
  \item Lemma \emph{closure\_map\_ball\_eq\_map\_closedball}
\begin{leancode}
lemma closure_map_ball_eq_map_closedball {n : ℕ} {j : hC.cell n} : 
  closure (hC.map n j '' ball 0 1) = hC.map n j '' closedBall 0 1
\end{leancode}
  \item Lemma \emph{not\_disjoint\_equal}
\begin{leancode}
lemma not_disjoint_equal {n : ℕ} {j : hC.cell n} {m : ℕ} {i : hC.cell m}
(notdisjoint: ¬ Disjoint (↑(hC.map n j) '' ball 0 1) (↑(hC.map m i) '' ball 0 1)) :
  (⟨n, j⟩ : (Σ n, hC.cell n)) = ⟨m, i⟩
\end{leancode}
  \item Lemma \emph{compact\_inter\_finite} \todo{Finish this. See Hatcher P.A.1}
\begin{leancode}
lemma compact_inter_finite (A : t.Compacts) : 
  _root_.Finite (Σ (m : ℕ), {j : hC.cell m // ¬ Disjoint A.1 (↑(hC.map m j) '' ball 0 1)})
\end{leancode}
  \item Lemma \emph{mapsto'}
\begin{leancode}
lemma mapsto' (n : ℕ) (i : hC.cell n) : ∃ I : Π m, Finset (hC.cell m),
  MapsTo (hC.map n i) (sphere 0 1) (⋃ (m < n) (j ∈ I m), hC.map m j '' ball 0 1)
\end{leancode}
  \item Lemma \emph{mapsto''} \todonoturgent{Do the proof.}
\begin{leancode}
lemma mapsto'' (n :  ℕ) (i : hC.cell n) : _root_.Finite (Σ (m : ℕ), 
{j : hC.cell m // ¬ Disjoint (↑(hC.map n i) '' sphere 0 1 ) (↑(hC.map m j) '' ball 0 1)} )
\end{leancode}
\end{itemize}

\section{File: Constructions}

\begin{itemize}
  \item Assumption:
\begin{leancode}
variable {X : Type*} [t : TopologicalSpace X] [T2Space X] {C : Set X} (hC : CWComplex C)
\end{leancode}
  \item Definition \emph{CWComplex\_level}
\begin{leancode}
def CWComplex_level (n : ℕ∞) : CWComplex (hC.level n) where
  cell l := {x : hC.cell l // l < n + 1}
  map l i := hC.map l i
  source_eq l i := sorry
  cont l i := sorry
  cont_symm l i := sorry
  pairwiseDisjoint := sorry
  mapsto l i := sorry
  closed A := sorry
  union := sorry
\end{leancode}
  \item Assumption \lean /variable {D : Set X} (hD : CWComplex D)/
  \item Definition \emph{CWComplex\_disjointUnion}
\begin{leancode}
def CWComplex_disjointUnion (disjoint : Disjoint C D) : CWComplex (C ∪ D) where
  cell n := Sum (hC.cell n) (hD.cell n)
  map n i :=
    match i with
    | Sum.inl x => hC.map n x
    | Sum.inr x => hD.map n x
  source_eq n i := sorry
  cont n i := sorry
  cont_symm n i := sorry
  pairwiseDisjoint := sorry
  mapsto n i := sorry
  closed A := sorry
  union := sorry
\end{leancode}
  \item Definition \emph{CWComplex\_subcomplex} \question{I think this should be an instance maybe?}
\begin{leancode}
def CWComplex_subcomplex (E : Set X) (subcomplex: Subcomplex hC E) : CWComplex E where
  cell n := subcomplex.I n
  map n i := hC.map n i
  source_eq n i := sorry
  cont n i := sorry
  cont_symm n i := sorry
  pairwiseDisjoint := sorry 
  mapsto n i := sorry
  closed A := sorry 
  union := sorry 
\end{leancode}
  \item Assumption 
\begin{leancode}
variable {X : Type*} {Y : Type*} [t1 : TopologicalSpace X] [t2 : TopologicalSpace Y]
[T2Space X] [T2Space Y] {C : Set X} {D : Set Y} (hC : @CWComplex X t1 C)
(hD : @CWComplex Y t2 D)
\end{leancode}
  \item Definition \emph{Prodkification}
\begin{leancode}
def Prodkification X Y := kification (X × Y)
\end{leancode}
  \item Notation \lean /×ₖ/
\begin{leancode}
infixr:35 " ×ₖ "  => Prodkification
\end{leancode}
  \item Definition \emph{CWComplex\_product} \todo{Make this work and do the proofs!}
\begin{leancode}
instance CWComplex_product : @CWComplex (X ×ₖ Y) instprodkification (C ×ˢ D) where
  cell n := (Σ' (m : ℕ) (l : ℕ) (hml : m + l = n), hC.cell m × hD.cell l)
  map n i := sorry
  source_eq n i := sorry
  cont n i := sorry
  cont_symm := sorry
  pairwiseDisjoint := sorry
  mapsto n i := sorry
  closed A := sorry
  union := sorry
\end{leancode}
\end{itemize}
\todo[inline]{Define Quotients.}

\section{File: Lemmas}


\begin{itemize}
  \item Assumption
\begin{leancode}
variable {X : Type*} [t : TopologicalSpace X] [T2Space X] {C : Set X} (hC : CWComplex C)
\end{leancode}
  \item Lemma \emph{isClosed\_level}
\begin{leancode}
lemma isClosed_level (n : ℕ∞) : IsClosed (hC.level n)
\end{leancode}
  \item Lemma \emph{isDiscrete\_level\_zero}
\begin{leancode}
lemma isDiscrete_level_zero {A : Set X} : IsClosed (A ∩ hC.level 0)
\end{leancode}
\todo[inline]{Make the following lemmata work with new definitions}
  \item Lemma \emph{iUnion\_subcomplex} \comment{Use CWComplex\_ subcomplex}\question{Again I need the indexed sum/disjoint union here.}
\begin{leancode}
lemma iUnion_subcomplex (J : Type u) (I : J → Π n, Set (hC.cell n))
(cw : ∀ (l : J), CWComplex (⋃ (n : ℕ) (j : I l n), hC.map n j '' ball 0 1)) 
  : CWComplex (⋃ (l : J) (n : ℕ) (j : I l n), hC.map n j '' ball 0 1)
\end{leancode}
  \item Lemma \emph{finite\_iUnion\_finitesubcomplex} \todo{Do the proofs.}
\begin{leancode}
lemma finite_iUnion_finitesubcomplex (m : ℕ) (I : Fin m → Π n, Set (hC.cell n))
(fincw : ∀ (l : Fin m), FiniteCWComplex (⋃ (n : ℕ) (j : I l n), hC.map n j '' ball 0 1)) : 
FiniteCWComplex (⋃ (l : Fin m) (n : ℕ) (j : I l n), hC.map n j '' ball 0 1) where
  cwcomplex := sorry
  finitelevels := sorry
  finitecells := sorry
\end{leancode}
  \item Definition \emph{open\_neighbourhood\_aux} \comment{See Hatcher p. 522. I don't really want to do that know so I'll just leave it here for now.} \todo{Make this work.}
\begin{leancode}
def open_neighbourhood_aux (ε : (n : ℕ) → (hC.cell n) → {x : ℝ // x > 0}) 
(A : Set X) (AsubC : A ⊆ C) (n : ℕ): Set X :=
  match n with
  | 0 => A ∩ hC.level 0
  | Nat.succ m => sorry
\end{leancode}
\end{itemize}
\end{document}
